\chapter{Analyseur de réseau et Antenne quart d'onde}

\section{But de la manipulation}

La première partie sert à comprendre, du point de vue de la mesure, le fonctionnement d'un
analyseur de réseau vectoriel. Cette partie concerne le calibrage de l'appareil par la méthode
dite du ``full-two-port''. Celle-ci ne peut s'effectuer que si l'on a bien saisi les causes d'erreurs
systématiques de la mesure.

La seconde partie consiste à mesurer les caractéristiques d'une antenne type ``dipôle quart
d'onde''. On s'intéressera plus précisément à son impédance d'entrée et à son gain. Au point de
vue de l'utilisation de l'analyseur de réseau, cela se traduira par la mesure d'un dipôle en
réflexion, et par la mesure d'un quadripôle en transmission.

\section{Rappel sur les erreurs systématiques}

Lorsqu'on mesure un dipôle, on utilise un montage à deux coupleurs:

\begin{figure}[H]
    \centering
    \includegraphics[width=0.5\textwidth]{TP_Antenne_Quart_Onde/fig/Montage_2_coupleurs.png}
\end{figure}

Si aucune erreur n'intervient, le rapport $\frac{b_1}{a_1}$
est proportionnel au coefficient de réflexion:

\begin{equation}
    \frac{b_1}{a_1}=E_{RF}\Gamma
    \label{eq_1}
\end{equation}


La mauvaise directivité du coupleur mesurant $b_1$ introduit un terme additif $E_{DF}$. 
L'impédance de Thévenin que voit le dispositif à mesurer n'est pas adaptée. Elle a un coefficient de réflexion
égal à $E_{SF}$. On a un régime de réflexions multiples:

\begin{equation}
    \frac{b_1}{a_1}=E_{DF}+\frac{E_{RF}\Gamma}{1-E_{SF}\Gamma}
    \label{eq_2}
\end{equation}

Il faut donc, en théorie, disposer de trois coefficients de réflexion étalons pour déterminer les
trois termes $E_{RF}$, $E_{DF}$, $E_{SF}$.

\newpage
Lorsqu'on effectue les mesures d'un quadripôle, on ajoute un terme de proportionnalité $E_{TF}$, un
terme de désadaptation de charge $E_{LF}$, et un terme d'isolation $E_{XF}$. Les équations linéaires entre
$a_1$, $b_1$ et $b_2$ sont représentées par un graphe de fluence:

\begin{figure}[H]
    \centering
    \includegraphics[width=0.8\textwidth]{TP_Antenne_Quart_Onde/fig/graph_Fluence.png}
\end{figure}

Lorsqu'on fait les mesures inverses, on intervertit, par des commutateurs internes, ce qui est
avant $R_1$, et ce qui est après $R_2$. Le graphe de fluence est alors:

\begin{figure}[H]
    \centering
    \includegraphics[width=0.75\textwidth]{TP_Antenne_Quart_Onde/fig/graph_Fluence2.png}
\end{figure}

\section{Préparation}

\subsection{Mesures en réflexion}

\begin{itemize}
    \item Interpréter physiquement l'équation (\ref{eq_1})
    \item \underline{\textbf{Charge coulissante}}: On place, en réflexion, une charge dont le $|\Gamma|$ est faible (de l'ordre
de 0.1), et dont on peut faire varier mécaniquement $Arg(\Gamma)$. En appliquant l'équation
(\ref{eq_2}), trouver, pour une fréquence fixe, le lieu de $\frac{b_1}{a_1}$ dans le plan complexe. Quel est le
terme d'erreur que l'on peut déduire de cette mesure?
\end{itemize}

\subsection{Mesures en transmission}

Le calibrage en réflexion consiste à effectuer les mesures lorsqu'on charge chaque accès par
une charge adaptée, puis lorsque l'on relie les deux accès. Expliquer comment, physiquement
(sans équation), on peut déterminer $E_{XF}$, $E_{LF}$, $E_{TF}$.

\section{Manipulations}

\subsection{Calibrage complet}

\begin{itemize}
    \item Mesurer un atténuateur court-circuité en réflexion et un atténuateur en transmission sans calibration du VNA\@.
    \item Effectuer le calibrage complet en réflexion et en transmission (full-two-port).
    \item Mesurer de nouveau l'atténuateur court-circuité en réflexion et l'atténuateur en transmission avec la calibration du VNA\@.
    \item Conclure et interpréter les défauts constatés sans correction.
\end{itemize}

\subsection{Mesure de l'antenne quart d'onde}

\subsubsection{Mesure en réflexion}
\begin{itemize}
    \item Déterminer un ordre de grandeur de la fréquence de résonance du dipôle quart d'onde proposé.
    \item Effectuer un calibrage complet au niveau du plan de masse. (Effectuer d'abord un calibrage en
réflexion sur connecteurs type SMA, puis faire un changement de plan de référence pour se
ramener au niveau de l'antenne). On choisira une bande de $\pm 20 \%$ de la fréquence de
résonnance.
    \item Mesurer $S_{11}$ pour une des antennes dont on notera la hauteur $h$ par rapport au plan de masse.
    \item Observer la résonance. S'agit-il d'une résonance série ou bien d'une résonance parallèle? Noter
la valeur de la fréquence de résonance $f_0$.
    \item Relever les courbes $R(f)$ et $X(f)$ (parties réelle et imaginaire de l'impédance d'entrée de l'antenne).
Donner la valeur du facteur de raccourcissement $K$:
\begin{equation*}
    K=\dfrac{\dfrac{\lambda_0}{4}-h}{\dfrac{\lambda_0}{4}}
\end{equation*}
\item Déterminer le coefficient de qualité du doublet, celui-ci peut être obtenu à l'aide de:
\begin{equation*}
    Q=\frac{1}{2}\frac{\partial X}{\partial f}\frac{f_0}{R}
\end{equation*}

\end{itemize}

\subsubsection{Mesure en transmission}

On place deux antennes quart d'onde identiques à la résonance pour établir une transmission
en espace libre. Mesurer l'affaiblissement de la transmission.

En déduire le gain d'un dipôle (tenir compte de la désadaptation). Comparer à la valeur théorique.