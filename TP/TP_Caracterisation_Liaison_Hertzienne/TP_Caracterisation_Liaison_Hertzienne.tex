\chapter{Caracterisation d'une liaison hertzienne}

\section*{Objectif du TP et présentation du banc de mesure}

Le but de ce TP est de mettre en évidence l'effet des réflexions par le sol.

Le banc de mesure est constituée par une antenne émettrice alimentée par une diode Gunn et une
antenne réceptrice connectée à un wattmètre

\begin{figure}[H]
    \centering
    \includegraphics[width=0.8\textwidth]{TP_Caracterisation_Liaison_Hertzienne/fig/banc_mesure.png}
    \includegraphics[width=0.7\textwidth]{TP_Caracterisation_Liaison_Hertzienne/fig/modelisation.png}
\end{figure}

\section{Préparation}

\begin{enumerate}
    \item Expliquer le phénomène physique que l'on doit observer sur la mesure de la puissance
reçue sur l'antenne de réception. Comment varie cette puissance en fonction de la distance
entre les antennes.
    \item En faisant l'hypothèse que la distance entre antennes est grande devant les hauteurs $h_1$ et
$h_2$ des antennes, montrer que le déphasage entre les deux trajets $r_1$ et $r_2$ est donné
approximativement par:

\[\Delta \phi \approx \frac{4\pi h_1 h_2}{\lambda d}\]

    \item En déduire que la distance qui sépare deux maxima ou deux minima successifs pour $h_1$ et
$h_2$ constants est donnée par:

\[\Delta d \approx \frac{\lambda d^2}{2 h_1 h_2}\]

\end{enumerate}

\section{Réglage du banc de mesure}

\begin{enumerate}
    \item Déterminer expérimentalement une valeur optimale de la fréquence de travail à l'aide de
la cavité située en amont de la diode Gunn.
    \item Déterminer la polarisation du champ.
    \item Mesurer le gain des antennes
\end{enumerate}

\section{Réflexion sur le sol}

Insérer une plaque métallique horizontale sur le banc de mesure.

\begin{enumerate}
    \item Tracer le niveau de sortie en fonction de la distance d entre les cornets
    \item Comparer la valeur de $\Delta d$ obtenue expérimentalement à celle obtenue théoriquement.
\end{enumerate}

\section{Simulation de la liaison hertzienne}

Ouvrir le logiciel ADS\@. Créer un nouveau workspace, et un nouveau schematic. Créer la liaison
caractérisée par le schéma suivant:

\begin{figure}[H]
    \centering
    \includegraphics[width=0.8\textwidth]{TP_Caracterisation_Liaison_Hertzienne/fig/Simulation.png}
\end{figure}

Le modèle de simulation comprend les éléments suivants

(sous la forme: nom\_librairie \ldots nom\_composant):

\begin{itemize}
    \item \textit{``Timed Sources \ldots ConstTimed''}: porteuse RF non modulée de niveau constant
    \begin{itemize}
        \item TStep = 10 secondes (période d'échantillonnage)
        \item FCarrier = mesurée en début de séance
        \item Value = 1.0
    \end{itemize}
    \newpage
    \item ``Antennas \& Propagation \ldots AntBase'': antenne d'émission (station de base)
    \begin{itemize}
        \item Gain = gain mesuré en début de séance
        \item X = Y = 0 (position de l'antenne dans le plan: à l'origine)
        \item Height = à mesurer
    \end{itemize}
    \item ``Antennas \& Propagation \ldots PropFlatEarth'': canal de propagation hertzien (modèle terrestre ``plat'' théorique)
    \begin{itemize}
        \item Polarization = à mesurer
        \item Permittivity, Conductivity: valeurs définies dans la suite
    \end{itemize}
    \item ``Antennas \& Propagation \ldots AntMobile'': antenne de réception mobile (ce modèle permet de faire
varier la distance entre les deux antennes)
    \begin{itemize}
        \item Gain = gain mesuré en début de séance
        \item X = X0 (position initiale de l'antenne de réception, définie dans le bloc ``Variables'')
        \item Y = 0
        \item Height = à mesurer
        \item SpeedType = km/h
        \item Vx = Vx (vitesse en km/h de l'antenne mobile, définie dans le bloc ``Variables'')
        \item Vy = 0 (déplacement suivant l'axes des x)
    \end{itemize}
    \item ``Sinks \ldots TimedSink'': prélèvement du signal de sortie
    \begin{itemize}
        \item Les paramètres par défaut sont corrects.
    \end{itemize}

    \vspace{0.25cm}
    Ajouter sur le schéma les blocs suivants:

    \item ``Controllers \ldots DF'': bloc de définition des paramètres de simulation, avec:
    \begin{itemize}
        \item Dans Controls: DefaultTimeStop = 1000 s
        \item Dans Options: OutVar = X0 Vx (récupération des variables à tracer; attention, séparateur = espace)
    \end{itemize}

    \item ``Controllers \ldots VAR eqn'': bloc de définition d'équations et de variables, avec:
    \begin{itemize}
        \item X0
        \item Vx
    \end{itemize}
\end{itemize}

\newpage
\textbf{\underline{Remarque}}

Le logiciel utilise le concept d'enveloppe complexe pour effectuer les simulations. Soit un signal
$x(t)$ de type passe-bande --- encore appelé signal bande étroite --- typiquement constitué par une
porteuse modulée, d'expression générale:

\[x(t) = i_x(t) \cos(2\pi f_c t) - q_x(t) \sin(2\pi f_c t)\]

ou

\[x(t)=\Re\{(i_x(t) + j q_x(t)) e^{j 2\pi f_c t}\}\]

On appelle \textit{enveloppe complexe} du signal $x(t)$ la grandeur $x_E(t)$ définie par: $i_x(t)+j q_x(t)$

L'enveloppe complexe $x_E(t)$ constitue un signal bande de base complexe associé au signal passe-bande
$x(t)$; il s'agit d'un processus beaucoup plus lent que la porteuse elle-même et transportant la
partie informationnelle du signal. On peut montrer que la simulation peut s'effectuer en calculant
uniquement l'enveloppe complexe, sans avoir à simuler la porteuse HF elle-même (un filtre passe-bande
est par exemple remplacé par un filtre passe-bas équivalent). Ceci permet de réduire les
durées de simulation dans des proportions considérables.

\textbf{\underline{Simulations}}

L'objectif est de relever, d'analyser et d'interpréter, la courbe du gain de liaison en fonction de la
distance dans les différentes situations. On tracera le gain en fonction de la distance.

\subsection{Banc de mesure}

Comparer les résultats de simulation aux résultats expérimentaux.

\subsection{Cas réaliste}

Utiliser les paramètres suivants: 
\begin{itemize}
    \item FCarrier = 6 GHz (fréquence porteuse)
    \item Gain = 40 dB (gain des antennes)
    \item Height = 20 m (antenne fixe) et 2 m (antenne mobile)
    \item vitesse véhicule = 100 km/h
    \item X0=1000
\end{itemize}

\begin{enumerate}
    \item Sol absent (permittivité relative: $\epsilon_r = 1$, conductivité: $\sigma = 0$)
    
    Comparer à la courbe théorique attendue (propagation en espace libre).

    \item Sol très bon conducteur (permittivité relative: $\epsilon_r = 1$, conductivité: $\sigma = 10^{10}$)
    
    Avec ces valeurs, le coefficient de réflexion sur le sol est très proche de 1.
    \begin{itemize}
        \item Déterminer les distances où un \textit{fading} apparaît. Comparer aux valeurs théoriques et
expérimentales.
        \item Superposer, sur le relevé, les dessins de la courbe théorique avec le sol absent et celui
de cette même courbe théorique translatée de +6 dB. Interpréter.
        \item Interpréter le comportement asymptotique de la courbe relevée.
    \end{itemize}
    \item Sol conducteur réel (permittivité relative: $\epsilon_r = 5$, conductivité: $\sigma = 100$)

    Avec ces valeurs, le coefficient de réflexion sur le sol est relativement proche de -1.
    \begin{itemize}
        \item Déterminer les distances où un \textit{fading} apparaît et comparer au cas précédent.
Comparer aux valeurs théoriques attendues.
        \item Interpréter le comportement asymptotique de la courbe relevée.
    \end{itemize}

\end{enumerate}