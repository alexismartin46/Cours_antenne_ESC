\documentclass[
10pt,
aspectratio=169,
]{beamer}


\usepackage{lipsum}
\usepackage{tikz}
\usepackage{bm}
\usepackage{tcolorbox}
\usepackage{beamerthemeensea}


\title{Antennes – ESC}
\subtitle{Cours 1: Equations de Maxwell et propagation des ondes}
\date{\the\year}
\author{Alexis MARTIN}
%\institute{ENSEA}

\usetheme{ensea}  


\begin{document}

\begin{frame}
\titlepage\end{frame}

\begin{frame}
\tableofcontents
\end{frame}

%%%%%%%%%%%%

\section{Equations de Maxwell}
\begin{frame} 
\frametitle{Equations de Maxwell} 

\begin{itemize}
\item Grandeurs instantannées: \hspace{1cm} $\vec{\bm{E}}(\vec{r},t),\vec{\bm{B}}(\vec{r},t),\vec{\bm{H}}(\vec{r},t),\vec{\bm{J}}(\vec{r},t),\vec{\bm{D}}(\vec{r},t), $

\item Equations de Maxwell: \\ 
\begin{minipage}{0.45\textwidth}
\begin{center} \begin{tcolorbox}[colframe=red,width=6.5cm]
\center$\vec{rot}(\vec{\bm{E}}(\vec{r},t))=-\dfrac{\partial\vec{\bm{B}}(\vec{r},t)}{\partial t}$\\
$\vec{rot}(\vec{\bm{H}}(\vec{r},t))=\vec{\bm{J}}(\vec{r},t)+\dfrac{\partial\vec{\bm{D}}(\vec{r},t)}{\partial t}$
\end{tcolorbox}\end{center}
\end{minipage}
\begin{minipage}{0.45\textwidth}
\begin{center} \begin{tcolorbox}[colframe=red,width=4cm]
\center$div(\vec{\bm{B}}(\vec{r},t))=0$\\
\vspace{0.5cm}
$div(\vec{\bm{E}}(\vec{r},t))=\dfrac{\rho}{\epsilon_0}$
\end{tcolorbox}\end{center}
\end{minipage}
\end{itemize}

\begin{minipage}[t]{0.45\textwidth}
\begin{itemize}

\item Conservation de la charge: \\
$div(\vec{\bm{J}}(\vec{r},t))+\dfrac{\partial\rho}{\partial t}=0$

\item Relations constitutives du milieu: 
\begin{itemize}
\item $\vec{\bm{D}}(\vec{r},t)=\epsilon_{3\times 3}\vec{\bm{E}}(\vec{r},t)$
\item $\vec{\bm{B}}(\vec{r},t)=\mu_{3\times 3}\vec{\bm{H}}(\vec{r},t)$
\item $\vec{\bm{J}}(\vec{r},t)=\sigma_{3\times 3}\vec{\bm{E}}(\vec{r},t)$

\end{itemize}
\end{itemize}
\end{minipage}
\begin{minipage}[t]{0.45\textwidth}
\begin{itemize}
    \item $\vec{\bm{E}}(\vec{r},t)$: champ électrique (V/m)
    \item $\vec{\bm{B}}(\vec{r},t)$: induction magnétique (T)
    \item $\vec{\bm{H}}(\vec{r},t)$: champ magnétique (A/m)
    \item $\vec{\bm{D}}(\vec{r},t)$: induction électrique (C/m$^2$)
    \item $\vec{\bm{J}}(\vec{r},t)$: densité de courant (A/m$^2$)
\end{itemize}

\end{minipage}

\end{frame}

%%%%%%%%%%%

\begin{frame} 
\frametitle{Equations de Maxwell} 

En régime sinusoïdal, découplage temps – position:

$\vec{\bm{E}}(\vec{r},t)=\vec{E_m}(\vec{r})\cos(\omega t + \varphi(\vec{r}))$

\vspace{1cm}
On associe $\vec{\bm{E}}(\vec{r},t)$ à sa représentation complexe $\vec{E}(\vec{r})$:

\begin{center} \begin{tcolorbox}[colframe=red,width=6cm]
$\vec{\bm{E}}(\vec{r},t)\rightarrow\vec{E}(\vec{r})=\vec{E_m}(\vec{r})e^{j\varphi(\vec{r})}$
\end{tcolorbox}\end{center}

\vspace{0.5cm}
Lorsque l'on utilise la représentation complexe, le temps n'intervient plus. 
Mais toutes les grandeurs ont une phase!
\end{frame}

%%%%%%%%%%%%

\begin{frame} 
\frametitle{Représentation d'un champ complexe} 

On peut uniquement représenter la partie réelle (composante en phase), et la partie imaginaire des champs (composante en quadrature).

Représentation en amplitude de $E_{phi}$ (symétrie cylindrique):

\begin{figure}
\includegraphics[width=0.35\textwidth]{figures/img_rep_champ_amp.png}\hspace{1cm}
\includegraphics[width=0.35\textwidth]{figures/img_rep_champ_amp2.png}
\end{figure}

\end{frame}

%%%%%%%%%%%%

\begin{frame} 
\frametitle{Représentation d'un champ complexe}

Représentation vectorielle:
\begin{figure}
\includegraphics[width=0.35\textwidth]{figures/img_rep_champ_vec.png}\hspace{1cm}
\includegraphics[width=0.35\textwidth]{figures/img_rep_champ_vec2.png}
\end{figure}

\end{frame}

%%%%%%%%%%%%

\begin{frame} 
\frametitle{Equations de Maxwell dans un milieu homogène et isotrope}
En milieu homogène et isotrope, les constantes caractéristiques du matériau ($\epsilon$, $\mu$, $\sigma$) sont indépendantes du point et de la direction d'observation. Les équations deviennent alors:

\begin{center} \begin{tcolorbox}[colframe=red,width=7cm]
\center$\vec{rot}(\vec{\bm{E}}(\vec{r},t))=-j\ \omega\ \mu\ \vec{\bm{H}}(\vec{r},t)$\\
$\vec{rot}(\vec{\bm{H}}(\vec{r},t))=\vec{\bm{J}}(\vec{r},t)+j\ \omega\ \epsilon\ \vec{\bm{E}}(\vec{r},t)$
\end{tcolorbox}\end{center}

Et en l'absence de sources (c'est à dire en l'absence de courant extérieur):
\begin{center}
$\vec{rot}(\vec{\bm{E}}(\vec{r},t))=-j\ \omega\ \mu\ \vec{\bm{H}}(\vec{r},t)$\\
$\vec{rot}(\vec{\bm{H}}(\vec{r},t))=j\ \omega\ \epsilon\ \vec{\bm{E}}(\vec{r},t)$
\end{center}

\end{frame}

%%%%%%%%%%%%

\begin{frame} 
\frametitle{Puissance ``rayonnée'' à travers une surface}

Soit $S_a$ une surface fermée. La puissance traversant cette surface est donnée par:
\begin{center} \begin{tcolorbox}[colframe=red,width=7cm]
\center\underline{Vecteur de Poynting}

$\vec{P}=\dfrac{1}{2}(\vec{E}\wedge\vec{H^*})$

$P=Re(\iint_{S_a} \vec{P}.\vec{dS})$
\end{tcolorbox}\end{center}

Cette formule ``fonctionne'' aussi pour des surfaces ouvertes. Elle permet, par exemple, de calculer la puissance injectée dans
un four à micro-ondes, en calculant cette intégrale surfacique, sur la surface de ``sortie'' du guide d'onde alimentant le four.

\vspace{0.5cm}
La même formule peut être utilisée en optique pour calculer, par exemple, la puissance transmise par un ``faisceau'' lumineux.


\end{frame}

%%%%%%%%%%%%

\section{Ondes planes}
\begin{frame} 
\frametitle{Ondes planes}

En l'absence de source (ou a une distance très grande des sources), les équations de Maxwell permettent de déduire l'équation de
propagation du champ électromagnétique:

\begin{center}
$\Delta \vec{E}-\epsilon.\mu\dfrac{\partial^2\vec{E}}{\partial t^2}=\vec{0}$ \hspace{1cm} et \hspace{1cm} $\Delta \vec{H}-\epsilon.\mu\dfrac{\partial^2\vec{H}}{\partial t^2}=\vec{0}$
\end{center}

Ces solutions particulières sont les \textbf{ondes planes} et peuvent être exprimées par:

\begin{center} \begin{tcolorbox}[colframe=red,width=7cm]
\center$\vec{E}(\vec{r})=\vec{E}(0)e^{-jk\vec{u}.\vec{r}}$

\vspace{0.5cm}
 $\vec{H}(\vec{r})=\vec{H}(0)e^{-jk\vec{u}.\vec{r}}$
\end{tcolorbox}\end{center}

$\vec{u}$ est un vecteur unitaire \hspace{1cm}$||\vec{u}||=1$

La constante $\vec{k}=k\vec{u}$ est appelée vecteur d'onde de l'onde plane.

Si $\vec{A}$ est une onde plane, on montre que:

\begin{center}
$div(\vec{A})=-j \vec{k}.\vec{A}$

$\vec{rot}(\vec{A})=-j \vec{k}\wedge\vec{A}$
\end{center}
\end{frame}

%%%%%%%%%%%%

\begin{frame}
\frametitle{Ondes planes: résultats remarquables}

$\vec{k}, \vec{E}$ et $\vec{H}$ forment une trièdre direct, avec:

\begin{minipage}[t]{0.45\textwidth}
\center\begin{tcolorbox}[colframe=red,width=4cm]
\center$\vec{H}=\dfrac{1}{\eta}(\vec{u}\wedge\vec{E})$

$\vec{E}=-\eta(\vec{u}\wedge\vec{H})$
\end{tcolorbox}
\end{minipage}
\hfill % Espacement entre les colonnes
\begin{minipage}[t]{0.45\textwidth}
\begin{tcolorbox}[colframe=red,width=4cm]
\center$\dfrac{||\vec{E}||}{||\vec{H}||}=\eta$
\end{tcolorbox}
\end{minipage}

\begin{minipage}[t]{0.45\textwidth}
\vspace{-0.9cm}La longueur d'onde $\lambda$ est donnée par:
\end{minipage}
\begin{minipage}[t]{0.35\textwidth}
\begin{tcolorbox}[colframe=red,width=4cm]
\center$k=\dfrac{2\pi}{\lambda}=\dfrac{\omega}{v}$
\end{tcolorbox}
\end{minipage}

\begin{minipage}[t]{0.45\textwidth}
\vspace{-0.9cm}L'impédance d'onde $\eta$ est donnée par:
\end{minipage}
\begin{minipage}[t]{0.35\textwidth}
\begin{tcolorbox}[colframe=red,width=4cm]
\center$\eta=\sqrt{\dfrac{\mu}{\epsilon}}$
\end{tcolorbox}
\end{minipage}

Dans le vide: $\eta=\sqrt{\dfrac{\mu_0}{\epsilon_0}}=377\ \Omega$ et $v=c=\dfrac{1}{\sqrt{\epsilon_0 . \mu_0}}=3.10^8\ m/s$.

\end{frame}

%%%%%%%%%%%%

\begin{frame}
\frametitle{Ondes planes: résultats remarquables}

La direction de propagation de l'onde plane est $\vec{u}$.

\vspace{0.5cm}
Les plans perpendiculaires à $\vec{u}$ sont des plans équiphase.

\vspace{0.5cm}
\begin{minipage}[t]{0.45\textwidth}
\vspace{-0.9cm}Le vecteur de Poynting est donné par:
\end{minipage}
\begin{minipage}[t]{0.35\textwidth}
\begin{tcolorbox}[colframe=red,width=4cm]
$\vec{P}=\dfrac{||\vec{E}||^2}{2\eta}\vec{u}$
\end{tcolorbox}
\end{minipage}

\vspace{0.5cm}
Une onde plane est \textbf{polarisée}: la direction de $\vec{E}$ est partout la même, et celle-ci est perpendiculaire à $\vec{u}$.

\vspace{0.5cm}
Le \textbf{plan de polarisation} en un point est défini par le plan $(\vec{u};\vec{E})$.

\end{frame}

%%%%%%%%%%%%

\begin{frame} 
\frametitle{Relation vecteur u $<->$ direction d'observation}

\begin{minipage}[t]{0.55\textwidth}
\begin{figure}
\includegraphics[width=\textwidth]{figures/demi_sphere.png}
\end{figure}
\end{minipage}
\hfill
\begin{minipage}[t]{0.35\textwidth}

\vspace{1.5cm}
$u_x=\sin(\theta)\cos(\varphi)$

\vspace{0.5cm}
$u_y=\sin(\theta)\sin(\varphi)$

\vspace{0.5cm}
$u_z=\cos(\theta)$
\end{minipage}

\end{frame}

%%%%%%%%%%%%
\section{Coordonnées cylindriques et sphériques}
\begin{frame} 
\frametitle{Coordonnées sphériques}

\begin{minipage}[t]{0.4\textwidth}
\begin{figure}
\includegraphics[width=\textwidth]{figures/coord_spherique.png}
\end{figure}
\end{minipage}
\hfill
\begin{minipage}[t]{0.55\textwidth}

$\vec{u}=\vec{e_r}$

\vspace{0.5cm}
$\vec{e_x}=\sin(\theta)\cos(\varphi)\vec{e_r}+\cos(\theta)\cos(\varphi)\vec{e_\theta}-\sin(\varphi)\vec{e_\varphi}$

\vspace{0.25cm}
$\vec{e_y}=\sin(\theta)\sin(\varphi)\vec{e_r}+\cos(\theta)\sin(\varphi)\vec{e_\theta}+\cos(\varphi)\vec{e_\varphi}$

\vspace{0.25cm}
$\vec{e_z}=\cos(\theta)\vec{e_r}-\sin(\theta)\vec{e_\theta}$

\vspace{0.5cm}
$\vec{e_r}=\sin(\theta)\cos(\varphi)\vec{e_x}+\sin(\theta)\sin(\varphi)\vec{e_y}+\cos(\theta)\vec{e_z}$

\vspace{0.25cm}
$\vec{e_\theta}=\cos(\theta)\cos(\varphi)\vec{e_x}+\cos(\theta)\sin(\varphi)\vec{e_y}-\sin(\theta)\vec{e_z}$

\vspace{0.25cm}
$\vec{e_\varphi}=-\sin(\varphi)\vec{e_x}+\cos(\varphi)\vec{e_y}$
\end{minipage}

\end{frame}

%%%%%%%%%%%%

\begin{frame} 
\frametitle{Coordonnées sphériques}

\center\begin{minipage}[t]{0.35\textwidth}
$\vec{grad}(U)=\begin{pmatrix}
\dfrac{\partial U}{\partial r} \vspace{0.25cm}\\
\dfrac{1}{r}\dfrac{\partial U}{\partial \theta} \vspace{0.25cm}\\
\dfrac{1}{r \sin\theta}\dfrac{\partial U}{\partial \varphi}
\end{pmatrix}$
\end{minipage}
\hfill
\begin{minipage}[t]{0.55\textwidth}

$\vec{rot}(\vec{A})=\begin{pmatrix}
\dfrac{1}{r \sin\theta}\dfrac{\partial}{\partial \theta}(\sin\theta A_\varphi)-\dfrac{1}{r \sin\theta}\dfrac{\partial A_\theta}{\partial\varphi} \vspace{0.25cm}\\
\dfrac{1}{r \sin\theta}\dfrac{\partial A_r}{\partial \varphi}-\dfrac{1}{r}\dfrac{\partial}{\partial r} (r A_\varphi)\vspace{0.25cm}\\
\dfrac{1}{r}\dfrac{\partial}{\partial r}(r A_\theta)-\dfrac{1}{r}\dfrac{\partial A_r}{\partial \theta}
\end{pmatrix}$

\end{minipage}

\vspace{0.5cm}
\[div(\vec{A})=\dfrac{1}{r^2}\dfrac{\partial}{\partial r}(r^2 A_r)+\dfrac{1}{r \sin\theta}\dfrac{\partial}{\partial \theta}(A_\theta \sin\theta)+\dfrac{1}{r\sin\theta}\dfrac{\partial A_\varphi}{\partial \varphi}\]

\vspace{0.5cm}
\[\Delta U=\dfrac{1}{r^2}\dfrac{\partial}{\partial r}(r^2 \dfrac{\partial U}{\partial r})+\dfrac{1}{r^2 \sin\theta}\dfrac{\partial}{\partial \theta}(\sin\theta\dfrac{\partial U}{\partial \theta})+\dfrac{1}{r^2 \sin^2\theta}\dfrac{\partial^2 U}{\partial \varphi^2}\]

\end{frame}

%%%%%%%%%%%%

\begin{frame}
\frametitle{Coordonnées cylindriques}

\center\begin{minipage}[t]{0.35\textwidth}
$\vec{grad}(U)=\begin{pmatrix}
\dfrac{\partial U}{\partial r} \vspace{0.25cm}\\
\dfrac{1}{r}\dfrac{\partial U}{\partial \theta} \vspace{0.25cm}\\
\dfrac{\partial U}{\partial z}
\end{pmatrix}$
\end{minipage}
\hfill
\begin{minipage}[t]{0.55\textwidth}

$\vec{rot}(\vec{A})=\begin{pmatrix}
\dfrac{1}{r}\dfrac{\partial A_z}{\partial \theta}-\dfrac{\partial A_\theta}{\partial z} \vspace{0.25cm}\\
\dfrac{\partial A_r}{\partial z}-\dfrac{\partial A_z}{\partial r}\vspace{0.25cm}\\
\dfrac{1}{r}\dfrac{\partial}{\partial r}(r A_\theta)-\dfrac{1}{r}\dfrac{\partial A_r}{\partial \theta}
\end{pmatrix}$

\end{minipage}

\vspace{0.5cm}
\[div(\vec{A})=\dfrac{1}{r}\dfrac{\partial}{\partial r}(r A_r)+\dfrac{1}{r}\dfrac{\partial A_\theta}{\partial \theta}+\dfrac{\partial A_z}{\partial z}\]

\vspace{0.5cm}
\[\Delta U=\dfrac{1}{r}\dfrac{\partial}{\partial r}(r \dfrac{\partial U}{\partial r})+\dfrac{1}{r^2}\dfrac{\partial^2 U}{\partial \theta^2}+\dfrac{\partial^2 U}{\partial z^2}\]

\end{frame}

%%%%%%%%%%%%
\section{Approximation en champ lointain}
\begin{frame} 
\frametitle{Approximation en champ lointain}

A l'infini, on peut montrer que les champs électriques et magnétiques sont reliés, localement, par:

\vspace{-0.5cm}
\[\vec{H}=\dfrac{1}{\eta}\vec{u}\wedge\vec{E}\]

On utilise le terme d'onde \textit{localement} plane, car même si cette relation est valable en chaque point, elle n'implique pas que la direction des champs soit la même partout.

\vspace{0.5cm}
\begin{minipage}[t]{0.65\textwidth}
En fait, les champs E et H sont tangents à une sphère centrée sur l'origine:

\vspace{0.5cm}
À direction constante, lorsque $r$ varie, l'amplitude des champs varie en $\dfrac{1}{r}$ et la phase comme celle de $e^{-jkr}$.

\end{minipage}
\begin{minipage}[t]{0.3\textwidth}
\vspace{-0.5cm}
\begin{figure}
\includegraphics[width=0.8\textwidth]{figures/sphere_champ_lointain.png}
\end{figure}
\end{minipage}

\vspace{-0.5cm}
L'approximation de champ lointaint est valable pour une distance 

\begin{minipage}[t]{0.2\textwidth}
\begin{tcolorbox}[colframe=red,width=2.5cm]
$r>\dfrac{2D^2}{\lambda}$
\end{tcolorbox}
\end{minipage}
\begin{minipage}[t]{0.75\textwidth}
\vspace{-1cm}
, où $D$ est la plus grande dimension de l'antenne.
\end{minipage}
\end{frame}

\end{document}

%%%%%%%%%%%%%%%%%%%%%%%%%%%%%%%%%%%%%%%%%%%%
%%%%%%%%% Annexes %%%%%%%%%%%%%%%%%%%%%%%%%%
%%%%%%%%%%%%%%%%%%%%%%%%%%%%%%%%%%%%%%%%%%%%


%%%%%%%%%%%%%%%

\begin{frame}
\frametitle{Ouverture plane}

Les sources sont du côté $ z < 0$, le demi-espace $ z > 0$ est vide.

\begin{figure}
\includegraphics[width=0.5\textwidth]{figures/ouverture_plane_schema.png}
\end{figure}

\begin{enumerate}
\item On calcule le Spectre d'Ondes Planes à l'aide de:

\begin{tcolorbox}[colframe=red,width=10cm]
$\vec{A}(u_x,u_y)=\dfrac{1}{\lambda^2} \displaystyle\iint_{R^2} \vec{E}(x',y',0)e^{+j\ k(x'u_x+y'u_y)}dx'\ dy' $
\end{tcolorbox}

\end{enumerate}

\end{frame}

%%%%%%%%%%%%

\begin{frame}
\frametitle{Ouverture plane}

Il s'agit d'une transformée de Fourier à 2D\@:

\begin{center}
$\vec{A}(u_x,u_y)=\dfrac{1}{\lambda^2} \displaystyle\iint_{R^2} \vec{E}(x',y',0)e^{+j k(x'u_x+y'u_y)}dx' dy' $
\end{center}

\begin{figure}
\includegraphics[width=0.5\textwidth]{figures/ouverture_plane_TF.png}
\end{figure}

\begin{minipage}[t]{0.45\textwidth}
Analogie avec la T.F. des physiciens

\vspace{0.25cm}
T.F. de Matematica
\end{minipage}
\begin{minipage}[t]{0.45\textwidth}
$F(x)=\displaystyle\int f(x)e^{j k_x x}dx$

$f(x)=\dfrac{1}{2\pi}\displaystyle\int F(k_x)e^{-j k_x x}dk_x$
\end{minipage}
\end{frame}

%%%%%%%%%%%%

\begin{frame} 
\frametitle{Ouverture plane}

\begin{enumerate}
\setcounter{enumi}{1}
\item On peut calculer tout champ du côté $z \geq  0$ à l'aide de:

\vspace{0.25cm}
\begin{center}
$\vec{E}(\vec{r})= \displaystyle\iint_{R^2} \vec{A}(u_x,u_y) e^{-j k(\vec{r}.\vec{u})}du_x du_y $

\vspace{0.25cm}
$\vec{H}(\vec{r})=\dfrac{1}{\eta} \displaystyle\iint_{R^2} \vec{u}\wedge\vec{A}(u_x,u_y) e^{-j k(\vec{r}.\vec{u})}du_x du_y $
\end{center}

\end{enumerate}

\vspace{0.25cm}
Remarques:
\begin{itemize}
\item L'expression ci-dessus est aussi valable pour un point $\vec{r}$ du plan $z = 0$. C'est la TF inverse de l'équation de la page précédente.

\item $\vec{A}(u_x,u_y)e^{-jk\vec{r}.\vec{u}}$ est l'expression d'une onde plane. Le champ en tout point ``en avant'' est donc exprimé comme la somme d'une infinité d'ondes planes.
\end{itemize}

\end{frame}

%%%%%%%%%%%%
\section{Approximation en champ lointain}
\begin{frame} 
\frametitle{Approximation en champ lointain}

Lorsque r tend vers l'infini dans la direction $\vec{u}_0$:

$\vec{r}_0=r \vec{u}_0=r\begin{pmatrix}
u_{x0}\\u_{y0}\\u_{z0}
\end{pmatrix}$, avec: $r \rightarrow \infty$

$\vec{E}(\vec{r}_0)=\displaystyle\iint_{R^2}\vec{A}(u_x,u_y)e^{-jkr(u_{x0}u_x+u_{y0}u_y+u_{z0}uz)}du_x du_y$

\vspace{0.25cm}
On montre que la limite est donnée par (\textbf{\underline{Principe de Huygens-Fresnel}}):

\begin{minipage}[t]{0.6\textwidth}
\begin{tcolorbox}[colframe=red,width=7.5cm]
$\vec{E}(\vec{r}_0) \underset{r \rightarrow \infty}{\approx} j \dfrac{e^{-jkr}}{r}\lambda u_{z0}\vec{A}(u_{x0},u_{y0})$
%$\vec{E}(\vec{r}_0) \underset{r \rightarrow \infty}{\approx} j \dfrac{e^{-jkr}}{r}\lambda \vec{A}(u_{x0},u_{y0})$

$\vec{H}(\vec{r}_0) \underset{r \rightarrow \infty}{\approx} \dfrac{1}{\eta} j \dfrac{e^{-jkr}}{r}\lambda u_{z0}\vec{u}_0\wedge\vec{A}(u_{x0},u_{y0})$
%$\vec{H}(\vec{r}_0) \underset{r \rightarrow \infty}{\approx} \dfrac{1}{\eta} j \dfrac{e^{-jkr}}{r}\lambda \vec{u}_0\wedge\vec{A}(u_{x0},u_{y0})$
\end{tcolorbox}
\end{minipage}
\begin{minipage}[t]{0.35\textwidth}
\vspace{-1cm}
Avec: $\vec{E}(\vec{r}_0) \perp \vec{H}(\vec{r}_0) \perp \vec{u}_0$
\end{minipage}

\end{frame}

%%%%%%%%%%%%

\begin{frame} 
\frametitle{Approximation en champ lointain}

\begin{minipage}[t]{0.6\textwidth}
\begin{tcolorbox}[colframe=red,width=7.5cm]
$\vec{E}(\vec{r}_0) \underset{r \rightarrow \infty}{\approx} j \dfrac{e^{-jkr}}{r}\lambda u_{z0}\vec{A}(u_{x0},u_{y0})$
%$\vec{E}(\vec{r}_0) \underset{r \rightarrow \infty}{\approx} j \dfrac{e^{-jkr}}{r}\lambda \vec{A}(u_{x0},u_{y0})$

$\vec{H}(\vec{r}_0) \underset{r \rightarrow \infty}{\approx} \dfrac{1}{\eta} j \dfrac{e^{-jkr}}{r}\lambda u_{z0}\vec{u}_0\wedge\vec{A}(u_{x0},u_{y0})$
%$\vec{H}(\vec{r}_0) \underset{r \rightarrow \infty}{\approx} \dfrac{1}{\eta} j \dfrac{e^{-jkr}}{r}\lambda \vec{u}_0\wedge\vec{A}(u_{x0},u_{y0})$
\end{tcolorbox}
\end{minipage}
\begin{minipage}[t]{0.35\textwidth}
\vspace{-1cm}
Avec: $\vec{E}(\vec{r}_0) \perp \vec{H}(\vec{r}_0) \perp \vec{u}_0$
\end{minipage}

\vspace{0.5cm}
En un point du champ lointain, le champ rayonné par une source quelconque a donc beaucoup de ressemblance avec une onde
plane.

\vspace{0.5cm}
$\vec{u}_0$ est la direction reliant l'antenne source avec le point d'observation.

\vspace{0.5cm}
\textbf{$\vec{u}_0$ est aussi la direction de propagation de cette onde localement plane.}

\end{frame}

%%%%%%%%%%%%

\begin{frame} 
\frametitle{Approximation en champ lointain}

\begin{minipage}[t]{0.6\textwidth}
\begin{tcolorbox}[colframe=red,width=7.5cm]
$\vec{E}(\vec{r}_0) \underset{r \rightarrow \infty}{\approx} j \dfrac{e^{-jkr}}{r}\lambda u_{z0}\vec{A}(u_{x0},u_{y0})$
%$\vec{E}(\vec{r}_0) \underset{r \rightarrow \infty}{\approx} j \dfrac{e^{-jkr}}{r}\lambda \vec{A}(u_{x0},u_{y0})$

$\vec{H}(\vec{r}_0) \underset{r \rightarrow \infty}{\approx} \dfrac{1}{\eta} j \dfrac{e^{-jkr}}{r}\lambda u_{z0}\vec{u}_0\wedge\vec{A}(u_{x0},u_{y0})$
%$\vec{H}(\vec{r}_0) \underset{r \rightarrow \infty}{\approx} \dfrac{1}{\eta} j \dfrac{e^{-jkr}}{r}\lambda \vec{u}_0\wedge\vec{A}(u_{x0},u_{y0})$
\end{tcolorbox}
\end{minipage}
\begin{minipage}[t]{0.35\textwidth}
\vspace{-1cm}
Avec: $\vec{E}(\vec{r}_0) \perp \vec{H}(\vec{r}_0) \perp \vec{u}_0$
\end{minipage}

\vspace{0.5cm}
Les variations du spectre d'onde plane $\vec{A}(u_{x0},u_{y0})$, selon les différentes directions de l'espace, sont pratiquement identiques à celles de $\vec{E}(r.\vec{u}_0)$ selon ces mêmes directions.

\vspace{0.5cm}
À direction constante, lorsque $r$ varie, l'amplitude des champs varie en $\dfrac{1}{r}$ et la phase comme celle de $e^{-jkr}$.

\end{frame}

%%%%%%%%%%%%

\begin{frame} 
\frametitle{Approximation en champ lointain}

Il résulte que, à l'infini, les champs électriques et magnétiques sont reliés, localement, par:

\[\vec{H}=\dfrac{1}{\eta}\vec{u}\wedge\vec{E}\]

On utilise le terme \textit{localement} plane, car même si cette relation est valable en chaque point, elle n'implique pas que la direction des champs soit la même partout.

\vspace{0.5cm}
En fait, les champs E et H sont tangents à une sphère centrée sur l'origine:

\begin{figure}
\includegraphics[width=0.2\textwidth]{figures/sphere_champ_lointain.png}
\end{figure}

\end{frame}

%%%%%%%%%%%%

% \begin{frame}
% \frametitle{Domaine visible}

% Les formules théoriques ci-dessus sont valables pour toute valeur de $u_{x0}$ et $u_{y0}$ telles que: $u_{x0}^2+u_{y0}^2 \leq 1$

% \begin{figure}
% \includegraphics[width=0.7\textwidth]{figures/Domaine_visible.png}
% \end{figure}

% Mais elles sont aussi valables pour:  $u_{x0}^2+u_{y0}^2 > 1$ ! !

% \end{frame}

%%%%%%%%%%%%

\begin{frame} 
\frametitle{Découplage entre composantes: Polarisation}

\vspace{-0.25cm}
\[\vec{E}(r\vec{u}) \underset{r \rightarrow \infty}{\approx} j \dfrac{e^{-jkr}}{\lambda\ r} u_{z}\displaystyle\iint_{R^2}   \vec{E}(x',y',0)\ e^{+jk(x'u_x+y'u_y)} dx' dy'\]
%\[\vec{E}(r\vec{u}) \underset{r \rightarrow \infty}{\approx} j \dfrac{e^{-jkr}}{\lambda\ r} \displaystyle\iint_{R^2}   \vec{E}(x',y',0)\ e^{+jk(x'u_x+y'u_y)} dx' dy'\]

\vspace{0.5cm}
Donc: 
\[E_x(r\vec{u})\underset{r \rightarrow \infty}{\approx} j \dfrac{e^{-jkr}}{\lambda\ r} u_{z}\displaystyle\iint_{R^2}   E_x(x',y',0)\ e^{+jk(x'u_x+y'u_y)} dx' dy\]
\[E_y(r\vec{u})\underset{r \rightarrow \infty}{\approx} j \dfrac{e^{-jkr}}{\lambda\ r} u_{z}\displaystyle\iint_{R^2}   E_y(x',y',0)\ e^{+jk(x'u_x+y'u_y)} dx' dy\]
\[E_{xy}(r\vec{u})\underset{r \rightarrow \infty}{\approx} j \dfrac{e^{-jkr}}{\lambda\ r} u_{z}\displaystyle\iint_{R^2}   E_{xy}(x',y',0)\ e^{+jk(x'u_x+y'u_y)} dx' dy\]
%\[E_x(r\vec{u})\underset{r \rightarrow \infty}{\approx} j \dfrac{e^{-jkr}}{\lambda\ r} \displaystyle\iint_{R^2}   E_x(x',y',0)\ e^{+jk(x'u_x+y'u_y)} dx' dy\]
%\[E_y(r\vec{u})\underset{r \rightarrow \infty}{\approx} j \dfrac{e^{-jkr}}{\lambda\ r} \displaystyle\iint_{R^2}   E_y(x',y',0)\ e^{+jk(x'u_x+y'u_y)} dx' dy\]
%\[E_{z}(r\vec{u})\underset{r \rightarrow \infty}{\approx} j \dfrac{e^{-jkr}}{\lambda\ r} \displaystyle\iint_{R^2}   E_{z}(x',y',0)\ e^{+jk(x'u_x+y'u_y)} dx' dy\]

Or, en champ lointain: $\vec{E}.\vec{u}=E_x u_x+E_y u_y+E_z u_z=0$

On a donc pour une ouverture polarisée parallèlement à Oy:

\[||\vec{E}(\vec{r})||^2=\dfrac{1}{{(\lambda\ r)}^2}(1-u_x^2) \left|\left|\displaystyle \iint_{R^2} E_y(x',y',0)\ e^{+jk(u_x x'+u_y y')}dx' dy'\right|\right|^2\]
\end{frame}

%%%%%%%%%%%%
\section{Exemples: Ouverture circulaire et rectangulaire}
\begin{frame} 
\frametitle{Ouverture circulaire $\Phi = 5 \lambda$}

\begin{center}
Diagramme de rayonnement en amplitude pour une ouverture polarisée selon $y$
\end{center}

\begin{figure}
\includegraphics[width=\textwidth]{figures/diag_ouverture_circ_5l.png}
\end{figure}

\end{frame}

%%%%%%%%%%%%

\begin{frame} 
\frametitle{Ouverture circulaire $\Phi = 5 \lambda$}
\begin{center}
\begin{minipage}[t]{0.4\textwidth}
\begin{figure}
\includegraphics[width=\textwidth]{figures/champ_ouverture_circ_5l.png}
\caption{Champ sur l'ouverture}
\end{figure}
\end{minipage}\hspace{1cm}
\begin{minipage}[t]{0.4\textwidth}
\begin{figure}
\includegraphics[width=\textwidth]{figures/champ_ray_ouverture_circ_5l.png}
\caption{Champ rayonné}
\end{figure}
\end{minipage}
\end{center}

\end{frame}

%%%%%%%%%%%%

\begin{frame} 
\frametitle{Ouverture rectangulaire}

\begin{center}
Diagramme de rayonnement en amplitude pour une ouverture polarisée selon Y
\end{center}

\begin{figure}
\includegraphics[width=\textwidth]{figures/diag_ouverture_rec.png}
\end{figure}

\end{frame}

%%%%%%%%%%%%

\begin{frame} 
\frametitle{Ouverture rectangulaire}

\begin{center}
\begin{minipage}[t]{0.4\textwidth}
\begin{figure}
\includegraphics[width=\textwidth]{figures/champ_ouverture_rec.png}
\caption{Champ sur l'ouverture}
\end{figure}
\end{minipage}\hspace{1cm}
\begin{minipage}[t]{0.4\textwidth}
\begin{figure}
\includegraphics[width=\textwidth]{figures/champ_ray_ouverture_rec.png}
\caption{Champ rayonné}
\end{figure}
\end{minipage}
\end{center}

\end{frame}


