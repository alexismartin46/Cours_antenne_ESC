\documentclass[
10pt,
aspectratio=169,
]{beamer}


\usepackage{lipsum}
\usepackage{tikz}
\usepackage{bm}
\usepackage{tcolorbox}
\usepackage{beamerthemeensea}


\title{Antennes – ESC}
\subtitle{Cours 2: Propriétés générales des antennes}
\date{\the\year}
\author{Alexis MARTIN}
%\institute{ENSEA}

\usetheme{ensea}  


\begin{document}

\begin{frame}
\titlepage\end{frame}

\begin{frame}
\tableofcontents
\end{frame}

%%%%%%%%%%%%

\section{Exemples de technologies}
\begin{frame} 
\frametitle{Exemples de technologies}
\framesubtitle{Antennes filaires}
\vspace{-0.3cm}
\begin{figure}
\includegraphics[width=0.8\textwidth]{figures/Ex_antennes_fil.png}
\end{figure}

\end{frame}

%%%%%%%%%%%%

\begin{frame} 
\frametitle{Exemples de technologies}
\framesubtitle{Antennes à ouvertures: guides}

\begin{figure}
\includegraphics[width=0.85\textwidth]{figures/Ex_antennes_ouv.png}
\end{figure}

\end{frame}

%%%%%%%%%%%%

\begin{frame} 
\frametitle{Exemples de technologies}
\framesubtitle{Antennes à ouvertures: cornet}

\begin{figure}
\includegraphics[width=0.65\textwidth]{figures/Ex_antennes_ouv2.png}
\end{figure}

\end{frame}

%%%%%%%%%%%%

\begin{frame} 
\frametitle{Exemples de technologies}
\framesubtitle{Antennes à ouvertures: réflecteur (paraboles)}

\begin{figure}
\includegraphics[width=0.75\textwidth]{figures/Ex_antennes_ouv3.png}
\end{figure}

\end{frame}

%%%%%%%%%%%%

\begin{frame} 
\frametitle{Exemples de technologies}
\framesubtitle{Antennes Patch}

\begin{figure}
\includegraphics[width=0.85\textwidth]{figures/Ex_antennes_patch.png}
\end{figure}

\end{frame}

%%%%%%%%%%%%

\begin{frame} 
\frametitle{Exemples de technologies}
\framesubtitle{Réseau d'antennes}

\begin{figure}
\includegraphics[width=0.8\textwidth]{figures/Ex_antennes_array.png}
\end{figure}

\end{frame}

%%%%%%%%%%%%
\section{Angle solide}
\begin{frame} 
\frametitle{Angle solide}

L'angle solide ``\textit{mesure}'' l'étendue d'un cône:

\begin{minipage}{0.45\textwidth}\center\begin{tcolorbox}[colframe=red,width=4.5cm]
\center$d\Omega=\dfrac{dA}{dr}=\sin \theta d \theta d \varphi$
\end{tcolorbox}
\end{minipage}
\begin{minipage}
{0.45\textwidth}
\begin{figure}
\includegraphics[width=0.8\textwidth]{figures/angle_solide.png}
\end{figure}
\end{minipage}

Pour l'espace entier: $\Omega=4\pi$ stéradians.

Pour un demi espace: $\Omega=2\pi$ stéradians.
\end{frame}

%%%%%%%%%%%%
\begin{frame} 
\frametitle{Puissance émise par stéradian}

Soit un cône dont le sommet est l'antenne d'émission. La puissance émise dans ce cône peut être mesurée. Cette puissance est indépendante de la distance à l'origine.

\vspace{0.5cm}
\textit{La puissance émise est conservée à l'intérieur d'un cône ``centré'' sur l'antenne.}

\vspace{0.5cm}
Ceci est dû au fait qu'à grande distance, les champs émis varient en $1/r$.

\vspace{0.5cm}
On peut alors définir $U(\theta,\varphi)$ la puissance émise par unité d'angle solide. Unité: W/strd

\center\begin{tcolorbox}[colframe=red,width=7cm]
\center$U(\theta,\varphi)=\dfrac{dP}{d\Omega}=\dfrac{r^2 ||\vec{E}(r,\theta,\varphi)||^2}{2\eta}$
\end{tcolorbox}

\end{frame}

%%%%%%%%%%%%
\section{Caractéristique de rayonnement}
\begin{frame}
\frametitle{Caractéristique de rayonnement}

La caractéristique de rayonnement, $r(\theta,\varphi)$ est la puissance émise par stéradian, normalisée par rapport à son niveau max:

\begin{tcolorbox}[colframe=red,width=7cm]
    \center$r(\theta,\varphi)=\dfrac{U(\theta,\varphi)}{U_{\max}}$
\end{tcolorbox}

Par conséquent:
$r_{\max}=1$

\vspace{0.5cm}
La valeur de $r$ peut aussi être indiquée en dB\@:

\begin{tcolorbox}[colframe=red,width=6cm]
    \center$r_{dB}(\theta,\varphi)=10 \log_{10} r_{lin}$
\end{tcolorbox}
\end{frame}

%%%%%%%%%%%%%
\begin{frame}
\frametitle{Représentation de $r(\theta,\varphi)$}
Une direction, pointant du côté des $z > 0$ peut être représentée par 2 quantités:

\vspace{0.5cm}
Soit $\theta$ et $\varphi$ 

Soit $u_x$ et $u_y$

\vspace{0.5cm}
$u_x=\sin(\theta)\cos(\varphi)$

$u_y=\sin(\theta)\sin(\varphi)$

\vspace{0.5cm}
D'où les différentes représentations graphique possibles, qui peuvent être combinées avec une représentation de $r(\theta,\varphi)$.

\end{frame}

%%%%%%%%%%%%%%
\begin{frame}
\frametitle{Représentation de $r(\theta,\varphi)$}
En coordonnées $u_x$, $u_y$, en 3D\@:

\begin{figure}
\includegraphics[width=\textwidth]{figures/Representation_3D.png}
\end{figure}
\end{frame}

%%%%%%%%%%%%%%
% \begin{frame}
% \frametitle{Représentation de $r(\theta,\varphi)$}
% En coordonnées $u_x$, $u_y$, en dB, et en 2D\@:

% \begin{minipage}{0.45\textwidth}
% \center\begin{figure}
% \includegraphics[width=\textwidth]{figures/Representation_2D_lin_cart.png}
% \end{figure}

% Linéaire
% \end{minipage}
% \begin{minipage}{0.45\textwidth}
% \center\begin{figure}
% \includegraphics[width=\textwidth]{figures/Representation_2D_dB_cart.png}
% \end{figure}

% dB
% \end{minipage}

% \end{frame}

%%%%%%%%%%%%%%
\begin{frame}
\frametitle{Représentation de $r(\theta,\varphi)$}
En coordonnées $\theta$, $\varphi$:

\begin{minipage}{0.45\textwidth}
\center\begin{figure}
\includegraphics[width=\textwidth]{figures/Representation_2D_dB_cart.png}
\end{figure}
\end{minipage}
\begin{minipage}{0.45\textwidth}
\center\begin{figure}
\includegraphics[width=\textwidth]{figures/Representation_2D_dB_polaire.png}
\end{figure}
\end{minipage}

\end{frame}

%%%%%%%%%%%%%%
\section{Angle d'ouverture}
\begin{frame}

\frametitle{Angle d'ouverture à mi-puissance}

La mi-puissance correspond à $r_{lin} = 0.5$, ou encore $r_{dB} = -3$

\vspace{0.5cm}
Cet angle est aussi appelé ``angle d'ouverture à -3dB''

\vspace{0.5cm}
\begin{figure}
\includegraphics[width=0.4\textwidth]{figures/angle_ouverture.png}
\end{figure}
\end{frame}

%%%%%%%%%%%%%%%
\begin{frame}
\frametitle{Angle solide d'ouverture}
Cette notion est moins utilisée, mais peut être très ``parlante'':

\[\Omega_{ouv}=\int\int_{espace}r(\theta,\varphi) d\Omega=\int_{\varphi=0}^{2\pi}\int_{\theta=0}^\pi r(\theta,\varphi)\sin(\theta)d\theta d\varphi\]

\vspace{1cm}
Pour une antenne ``omni-directionnelle'', appelée aussi ``source isotrope'':

\[\Omega_{ouv}=4\pi\]

\end{frame}

%%%%%%%%%%%%%%%
\begin{frame}
\frametitle{Angle solide d'ouverture}

Pour une antenne avec un seul lobe principal et un faible niveau de lobes secondaires, 
assimilée à une ouverture à éclairement gaussien, en notant $\theta_H$ et 
$\theta_E$, les angles d'ouverture dans 2 plans perpendiculaires entre eux:

\vspace{0.5cm}
\begin{tcolorbox}[colframe=red,width=5cm]
\center$\Omega\approx\dfrac{\pi}{\ln 2}\sin\dfrac{\theta_H}{2}\sin\dfrac{\theta_E}{2}$
\end{tcolorbox}

\vspace{0.5cm}
Cette approximation est à privilégier pour une antenne \textbf{à faible
niveau de lobes secondaires}

\vspace{0.5cm}
Si $\theta_H$ et $\theta_E$ sont très petits, on a une approximation encore plus simple:

\begin{tcolorbox}[colframe=red,width=9cm]
    \center$\Omega\approx\dfrac{\pi}{\ln 2}\sin\dfrac{\theta_H}{2}\sin\dfrac{\theta_E}{2}\approx\dfrac{\pi}{4\ln 2}\theta_H\theta_E\approx\theta_H^{rad}\theta_E^{rad}$
    \end{tcolorbox}

\end{frame}

%%%%%%%%%%%%%%%%
\section{Directivité, efficacité et gain}
\begin{frame}
\frametitle{Directivité}

La directivité ``mesure'' le caractère ``pointu'' du diagramme de rayonnement.

\vspace{0.5cm}
\begin{tcolorbox}[colframe=red,width=8cm]
\center$D=\dfrac{U_{\max}}{U_{moyen}}=\dfrac{4\pi}{\int\int_{espace}r(\theta,\varphi)d\Omega}=\dfrac{4\pi}{\Omega_{ouv}}$
\end{tcolorbox}


\vspace{0.5cm}
\begin{tcolorbox}[colframe=red,width=2.5cm]
\center$D\geq 1$
\end{tcolorbox}

\vspace{0.5cm}
Pour une antenne omni-directionnelle: D = 1

\end{frame}

%%%%%%%%%%%%%%%%
\begin{frame}
\frametitle{Directivité}

Pour une antenne avec un faible niveau de lobes secondaires, et en
l'absence de ``rayonnement arrière'':

\vspace{0.5cm}
\begin{tcolorbox}[colframe=red,width=5cm]
\center$D\approx\dfrac{4\ln 2}{\sin\dfrac{\theta_H}{2}\sin\dfrac{\theta_E}{2}}$
\end{tcolorbox}

\vspace{0.5cm}

Approximation si l'antenne est assez directive:

\begin{tcolorbox}[colframe=red,width=3.5cm]
\center$D\approx\dfrac{36400}{\theta_H^\circ\theta_E^\circ}$
\end{tcolorbox}

\end{frame}

%%%%%%%%%%%%%%%%
\begin{frame}
\frametitle{Directivité \& Réciprocité}

\begin{tcolorbox}[colframe=red,width=5cm]
\center$D_{dB}=10\log_{10}D_{lin}$
\end{tcolorbox}

%\vspace{0.5cm}
Une liaison entre 2 antennes est réciproque:

\begin{minipage}{0.45\textwidth}
\begin{figure}
\includegraphics[width=\textwidth]{figures/reciproque.png}
\end{figure}
\end{minipage}
\begin{minipage}{0.45\textwidth}
\center{La propriété de \textit{réciprocité} est toujours vraie:}
\begin{tcolorbox}[colframe=red,width=3cm]\center$\dfrac{P_2}{P_1}=\dfrac{P_2'}{P_1'}$
\end{tcolorbox}
\end{minipage}

\end{frame}
%%%%%%%%%%%%%%%%
\begin{frame}
\frametitle{Efficacité d'une antenne}

\begin{figure}
\includegraphics[width=0.6\textwidth]{figures/efficacite.png}
\end{figure}

La très grande majorité des antennes ont un rendement très proche de 1, ce qui
revient à dire que la puissance totale rayonnée est égale à la puissance
absorbée par l'antenne.

\vspace{0.25cm}
Seules les petites antennes (petites devant la longueur d'onde) font
exception. Elles ont généralement un très mauvais rendement.

\vspace{0.25cm}
L'habitude est désormais de nommer cette grandeur \textit{efficacité d'une
antenne}
\begin{center}
\begin{tcolorbox}[colframe=red,width=5.5cm]
\center$\alpha=\dfrac{P_{ray}}{P_{abs}}=\text{efficacité}$
\end{tcolorbox}
\end{center}

\end{frame}

%%%%%%%%%%%%%%%
\begin{frame}
\frametitle{Gain d'une antenne}

Le gain d'une antenne permet de calculer la puissance (par unité
d'angle solide) dans la direction du maximum, à partir de la
puissance absorbée par l'antenne.

\begin{minipage}{0.45\textwidth}
\begin{tcolorbox}[colframe=red,width=4cm]
\center$G=\dfrac{U_{\max}}{\left(\dfrac{P_{abs}}{4\pi}\right)}$
\end{tcolorbox}
\end{minipage}
\begin{minipage}{0.45\textwidth}
\begin{tcolorbox}[colframe=red,width=5cm]
\center$G_{dB}=10\log_{10}G_{lin}$
\end{tcolorbox}
\end{minipage}

\vspace{0.5cm}
On trouve aussi souvent la notion de ``Gain réalisé'':

\begin{center}
\begin{tcolorbox}[colframe=red,width=6cm]
\center$G_{realized}=\dfrac{U_{\max}}{\left(\dfrac{P_{incidente}}{4\pi}\right)}$
\end{tcolorbox}

\begin{tcolorbox}[colframe=red,width=5cm]
\center$G_{realized}=(1-||\Gamma||^2) \ G$
\end{tcolorbox}
\end{center}



\end{frame}

%%%%%%%%%%%%%%%%
\begin{frame}
\frametitle{Relation Gain – Directivité}

\begin{tcolorbox}[colframe=red,width=4cm]
\center$D=\dfrac{U_{\max}}{\left(\dfrac{P_{ray}}{4\pi}\right)}$
\end{tcolorbox}

\vspace{0.5cm}
On a donc:
\begin{tcolorbox}[colframe=red,width=5cm]
\center$G=\alpha \ D$
\end{tcolorbox}

\vspace{0.5cm}
Le gain d'une antenne, et sa directivité ont donc généralement
des valeurs très proches.

\vspace{0.5cm}
Ces valeurs sont identiques pour une efficacité de 100 \%.

\end{frame}

%%%%%%%%%%%%%%%%
\begin{frame}
\frametitle{Surface de captation}

Cette notion s'applique à une antenne utilisée en réception. Cette
valeur est équivalent à la valeur de la ``surface effective'' d'un
capteur, en optique.

\vspace{0.5cm}
Soit $I(W / m^2)$, l'intensité de l'onde éclairant l'antenne.

Soit $P_{available}$, la puissance disponible aux bornes de l'antenne.

Soit $\alpha$, l'efficacité.

La surface de captation $S_c$ de l'antenne est donnée par:
\[S_c=\dfrac{P_{available}}{\alpha I}\]

\vspace{0.5cm}
On montre aussi que l'on a la relation suivante entre $S_c$ et $D$:
\begin{center}
\begin{tcolorbox}[colframe=red,width=3cm]
\center$S_c=\dfrac{\lambda^2}{4\pi}D$
\end{tcolorbox}
\end{center}

\end{frame}

%%%%%%%%%%%%%%%%
\section{Résistance de rayonnement}
\begin{frame}
\frametitle{Résistance de rayonnement}
Soit $I_{RMS}$, la valeur efficace du courant à l'entrée de l'antenne. La
résistance de rayonnement est définie par:

\begin{tcolorbox}[colframe=red,width=4cm]
\center$R_r=\dfrac{P_{ray}}{I_{RMS}^2}$
\end{tcolorbox}

\vspace{0.5cm}
Schéma électrique équivalent à l'entrée de l'antenne:

\begin{figure}
\includegraphics[width=0.5\textwidth]{figures/resistance_rayonnement.png}
\end{figure}

\end{frame}

%%%%%%%%%%%%%%%%
\begin{frame}
\frametitle{Résistance de rayonnement}
La puissance totale rayonnée $P_{ray}$ est donnée par:

\begin{tcolorbox}[colframe=red,width=8cm]
\center$P_{ray}=\int\int_{espace}U(\theta,\varphi)d\Omega=R_r.I_{RMS}^2$
\end{tcolorbox}

\vspace{0.5cm}
$R_p$ est la résistance de pertes de l'antenne (pertes par effet Joule, ou
pertes diélectriques)

\vspace{0.5cm}
$jX$ est la partie réactive de l'impédance de
l'antenne, nécessairement présente, et qui sera étudiée
ultérieurement.

\vspace{0.5cm}
L'efficacité est donnée par:

\begin{tcolorbox}[colframe=red,width=4cm]
\center$\alpha=\dfrac{R_r}{R_r+R_p}$
\end{tcolorbox}

\end{frame}

%%%%%%%%%%%%%%%%
\section{Bilan de liaison}
\begin{frame}
\frametitle{Bilan de liaison}

\begin{figure}
\includegraphics[width=0.8\textwidth]{figures/bilan_liaison.png}
\end{figure}

Soit une liaison entre 2 antennes. Chaque antenne ``pointe'' vers
l'autre dans la direction repérée par $\theta_i$, $\varphi_i$.

On note $P_2$ la puissance reçue et $P_1$ la puissance émise, et l'on
suppose que les antennes sont parfaitement adaptées. Alors:

\begin{center}
\begin{tcolorbox}[colframe=red,width=8cm]
\center$\dfrac{P_2}{P_1}=G_1.G_2.{\left(\dfrac{\lambda}{4\pi d}\right)}^2.r_1(\theta_1,\varphi_1).r_2(\theta_2,\varphi_2).\rho_p$
\end{tcolorbox}
\end{center}

\end{frame}

%%%%%%%%%%%%%%%%
\begin{frame}
\frametitle{Bilan de liaison}
$r_1(\theta_1, \varphi_1)$ et $r_2(\theta_2, \varphi_2)$ sont les caractéristiques de rayonnement des
antennes (coefficient positifs inférieurs ou égaux à 1)

$\rho_p$ est le facteur de polarisation. Ce coefficient est aussi un
coefficient positif inférieur ou égal à 1.

Le cas le plus favorable, que l'on choisit autant que possible, est
celui où chaque antenne ``pointe'' vers l'autre, avec $\rho_p$ = 1

\vspace{0.5cm}
Dans ce cas:
\begin{center}
\begin{tcolorbox}[colframe=red,width=5cm]
\center$\dfrac{P_2}{P_1}=G_1.G_2.{\left(\dfrac{\lambda}{4\pi d}\right)}^2$
\end{tcolorbox}


Cette formule est appelée ``\textit{Formule de Friis}''

\end{center}
\end{frame}

%%%%%%%%%%%%%
\begin{frame}
\frametitle{Cas d'antennes désadaptées}
Dans le cas où les antennes sont désadaptées, 
la formule précédente devient de façon évidente:

\begin{center}
\begin{tcolorbox}[colframe=red,width=9cm]
\center$\dfrac{P_2}{P_1}=G_1.G_2.{\left(\dfrac{\lambda}{4\pi d}\right)}^2 {(1-||\Gamma_1||)}^2.{(1-||\Gamma_2||)}^2$
\end{tcolorbox}
\end{center}

\vspace{0.5cm}
Ou alors, en faisant apparaître les ``gains réalisés'':

\begin{center}
\begin{tcolorbox}[colframe=red,width=5cm]
\center$\dfrac{P_2}{P_1}=G_{r1}.G_{r2}.{\left(\dfrac{\lambda}{4\pi d}\right)}^2$
\end{tcolorbox}
\end{center}

\end{frame}

%%%%%%%%%%%%%%
\begin{frame}
\frametitle{Désadaptation correspondant à un T.O.S. de 2}
Définition du taux d'Ondes Stationnaires:

\begin{tcolorbox}[colframe=red,width=4cm]
\center$T.O.S.=\dfrac{1+||\Gamma||}{1-||\Gamma||}$
\end{tcolorbox}

On a par exemple pour $||\Gamma||=\dfrac{1}{3}$, soit $||\Gamma||_{dB}=-9.54$ dB, soit un $T.O.S. = 2$:

\[{(1-||\Gamma||)}^2=-0.51\ dB\]

\begin{tcolorbox}[colframe=red,width=10cm]
    Un T.O.S. de 2 se traduit donc par une perte en transmission de 0,5 dB.
\end{tcolorbox}

\vspace{0.5cm}
Un coefficient de réflexion inférieur à -10 dB (T.O.S. = 2) est
souvent considéré comme suffisant pour une antenne.

\end{frame}

%%%%%%%%%%%%%%%%
\begin{frame}
    \frametitle{Polarisation circulaire}

La somme de 2 ondes polarisées perpendiculairement entre elles, de
même amplitude, et déphasées de $\pi/2$ donne une onde à polarisation
circulaire:

\begin{center}
\begin{tcolorbox}[colframe=red,width=6cm]
\center$\vec{E}=E_0\left(\cos(\omega t)\vec{e_x}+\sin(\omega t)\vec{e_y}\right)$
\end{tcolorbox}
\end{center}

\vspace{0.5cm}
La direction du champ électrique tourne en fonction du temps,
à la fréquence de l'onde.

Dans le cas précis, si la composante en phase du champ est portée
par Ox, alors la composante en quadrature (de phase) est portée par
Oy.

\end{frame}

%%%%%%%%%%%%%%%%
\section{Polarisation}
\begin{frame}
\frametitle{Polarisation circulaire}
Si, en un point donné, la composante sur y est en retard de $\pi/2$ sur la
composante sur x, alors le champ électrique tourne comme indiqué sur
la figure, et l'on est en \textbf{polarisation circulaire droite}:

\begin{figure}
\includegraphics[width=0.8\textwidth]{figures/polar_circ_droit.png}
\end{figure}

\end{frame}

%%%%%%%%%%%
\begin{frame}
\frametitle{Coefficient de polarisation (polarisation linéaire)}

Cas de 2 antennes à polarisation linéaire:

\begin{figure}
\includegraphics[width=0.8\textwidth]{figures/double_polar_lin.png}
\end{figure}

$E_1$ et $E_2$ conservent la même direction, tout le long de la ligne qui
relie les 2 antennes. Dans ce cas, si on note $\Psi$, l'angle entre les 2
plans de polarisation des antennes:

\begin{center}
\begin{tcolorbox}[colframe=red,width=4cm]
\center$\rho_p=\cos^2(\Psi)$
\end{tcolorbox}
\end{center}

\end{frame}

%%%%%%%%%%
\begin{frame}
\frametitle{Coefficient de polarisation (polarisation circulaire)}
Pour une onde à polarisation circulaire, les composantes en phase et
en quadrature sont (géométriquement) perpendiculaires entre elles.

\vspace{0.5cm}
\begin{itemize}
    \item Cas de 2 antennes à polarisation circulaire droite: $\rho_P = 1$
    \item Cas de 2 antennes à polarisation circulaire gauche: $\rho_P = 1$
    \item Une droite et une gauche: $\rho_P = 0$
    \item Une antenne à polarisation linéaire et une
antenne à polarisation circulaire: $\rho_P = \dfrac{1}{2}$
\end{itemize}

\end{frame}

%%%%%%%%%%%%%%%%
\section{Bruit thermique}
\begin{frame}
\frametitle{Rayonnement de corps noirs}

\begin{figure}
\includegraphics[width=0.6\textwidth]{figures/rayonnement_corps_noir.png}
\end{figure}

\end{frame}

%%%%%%%%%%%%%%%%
\begin{frame}
\frametitle{Bruit reçu par une antenne}

Soit une antenne, totalement entourée d'une ``cible'', à la
température $T_{cible}$:

\begin{minipage}{0.6\textwidth}
\begin{figure}
\includegraphics[width=\textwidth]{figures/antenne_bruit.png}
\end{figure}
\end{minipage}
\begin{minipage}{0.35\textwidth}
La puissance de bruit reçue est donnée par:

\begin{tcolorbox}[colframe=red,width=5cm]
\center$P_{bruit}=k.T_{cible}.\Delta f$
\end{tcolorbox}
\end{minipage}

\vspace{0.5cm}

(On est du côté droit de la figure précédente, où le bruit thermique
est proportionnel à la température)

\end{frame}

%%%%%%%%%%%%%%%%
\begin{frame}
\frametitle{Bruit reçu par une antenne}

Si l'antenne est entourée de plusieurs objets à des températures
différentes:

\begin{minipage}{0.5\textwidth}
\begin{figure}
\includegraphics[width=\textwidth]{figures/antenne_bruit2.png}
\end{figure}
\end{minipage}
\begin{minipage}{0.45\textwidth}
La puissance de bruit reçue est donnée par:
\[P'_N=k.T_A.\Delta f\]
où $T_A$ représente une ``température de bruit moyennée''.
\end{minipage}

Pour une antenne terrestre très directive pointée vers un satellite,
$T_A$ est très faible:

\[T_A\approx 10 \ K\]

\end{frame}

%%%%%%%%%%%%%%%%
\begin{frame}
\frametitle{Bruit reçu par une antenne}
Avec un pré-amplificateur connecté à l'antenne:

\begin{figure}
\includegraphics[width=0.6\textwidth]{figures/antenne_preampli.png}
\end{figure}

$F$ étant le facteur de bruit du pré-amplificateur, et $T_0$ la température
de référence \\(en général 290 K)

\vspace{0.5cm}
Température totale de bruit ramenée à l'entrée:

\begin{center}
\begin{tcolorbox}[colframe=red,width=7cm]
\center$T_{\text{totale ramenée}}=T_A+(F-1).T_0$
\end{tcolorbox}
\end{center}



\end{frame}

\end{document}