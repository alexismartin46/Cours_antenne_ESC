\documentclass[
10pt,
aspectratio=169,
]{beamer}


\usepackage{lipsum}
\usepackage{tikz}
\usepackage{bm}
\usepackage{tcolorbox}
\usepackage{beamerthemeensea}


\title{Antennes – ESC}
\subtitle{Cours 4: Antennes à ouverture}
\date{\the\year}
\author{Alexis MARTIN}
%\institute{ENSEA}

\usetheme{ensea}  


\begin{document}

\begin{frame}
\titlepage\end{frame}

\begin{frame}
\tableofcontents
\end{frame}

%%%%%%%%%%%%

\begin{frame} 
\frametitle{Exemples d'antennes à ouverture}

\begin{minipage}{0.45\textwidth}
    \centering\includegraphics[width=0.7\textwidth]{figures/antenne_guide.png}
\end{minipage}
\begin{minipage}{0.45\textwidth}
    \centering\includegraphics[width=0.7\textwidth]{figures/antenne_cornet.png}
\end{minipage}

\begin{minipage}{0.45\textwidth}
    Les antennes comportant un réflecteur
font partie de cette catégorie
d'antennes:
\end{minipage}
\begin{minipage}{0.45\textwidth}
    \centering\includegraphics[width=0.7\textwidth]{figures/antenne_reflecteur.png}
\end{minipage}


\end{frame}

%%%%%%%%%%%%
\section{Courants magnétiques}
\begin{frame}
\frametitle{Courants magnétiques}
On pourrait supposer, par souci de symétrie, l'existence de courants
magnétiques:

\begin{center}
    $\vec{M}$: densité de courant magnétique

    $\tau$: densité de charge magnétique
\end{center}

Avec ces courants magnétiques, les équations de Maxwell
deviendraient:

\[\vec{rot}(\vec{E})=-\vec{M}-j.\omega.\mu.\vec{H}\]

\[\vec{rot}(\vec{H})=\vec{J}+j.\omega.\varepsilon.\vec{E}\]
\end{frame}

%%%%%%%%%%%%%
\begin{frame}
    \frametitle{Symétrie E $<->$ H et J $<->$ M}
\begin{tabular}{c l}
    Partant de: & $\vec{rot}(\vec{E_1})= - j.\omega.\mu.\vec{H_1}$\\
    & $\vec{rot}(\vec{H_1})=\vec{J_1}+j.\omega.\varepsilon.\vec{E_1}$
\end{tabular}

\vspace{0.5cm}
\begin{tabular}{l}
Et en faisant les changements de variables:\\

$\vec{E_1}=\eta.\vec{H_2}$\hspace{0.5cm};\hspace{0.5cm}$\vec{H_1}=-\frac{\vec{E_2}}{\eta}$\hspace{0.5cm};\hspace{0.5cm}$\vec{J_1}=\dfrac{\vec{M_2}}{\eta}$
\end{tabular}

\vspace{0.5cm}
\begin{tabular}{c l}
    On obtient: & $\vec{rot}(\vec{E_2})= \vec{M_2} - j.\omega.\mu.\vec{H_2}$\\
    & $\vec{rot}(\vec{H_2})=j.\omega.\varepsilon.\vec{E_2}$
\end{tabular}

\vspace{0.25cm}
\begin{center}
\begin{tcolorbox}[colframe=red,width=11cm]
    Des courants magnétiques peuvent donc créer des champs
    identiques à ceux créés par une répartition de courant
    électrique, à condition d'échanger les champs E et H
\end{tcolorbox}
\end{center}

\end{frame}

%%%%%%%%%%%%%
\begin{frame}
\frametitle{Potentiel vecteur magnétique}
Comme $\vec{rot}(\vec{H_2})=j.\omega.\varepsilon.\vec{E_2}$ alors, $div(\vec{rot}(\vec{H_2}))=div(j.\omega.\varepsilon.\vec{E_2})=0$.

On a donc $div(\vec{E_2})=0$.

De même que pour les courants électriques, on peut définir un potentiel vecteur magnétique $\vec{F}$ pour des courants magnétiques tel que: $\vec{D}=\vec{rot}(\vec{F})$

\vspace{0.5cm}
L'equation de propagation du potentiel vecteur magnétique est:
\[\Delta \vec{F}+k^2\dfrac{\partial^2\vec{F}}{\partial t^2}=-\epsilon\vec{M}\]

\begin{minipage}[t]{0.45\textwidth}
\vspace{0.5cm}
La solution de cette équation est:

\begin{tcolorbox}[colframe=red,width=7cm]
\centering
$\vec{F}(\vec{r})=\dfrac{\epsilon}{4\pi}\iiint_{sources} \vec{M}(\vec{r'})\dfrac{e^{-jkR}}{R}dV'$
\end{tcolorbox}
\end{minipage}
\begin{minipage}[t]{0.45\textwidth}
\vspace{-0.5cm}
\raggedleft\includegraphics[width=0.7\textwidth]{figures/calcul_rayonnement.png}
\vspace{-0.3cm}
\end{minipage}

\end{frame}

%%%%%%%%%%%%%
\begin{frame}
\frametitle{Rayonnement d'une source quelconque de courants 
magnétiques}

On note $\Psi(\vec{r})=\dfrac{e^{-jk||\vec{r}||}}{||\vec{r}||}$

\begin{tcolorbox}[colframe=red,width=7.5cm]
\centering
$\vec{F}(\vec{r})=\dfrac{\epsilon}{4\pi}\iiint_{sources}\vec{M}(\vec{r'})\Psi(\vec{r}-\vec{r'})dV'$
\end{tcolorbox}

On en déduit le champ magnétique rayonné:

\begin{tcolorbox}[colframe=red,width=6.5cm]
\centering
$\vec{H}=\dfrac{1}{j\omega\epsilon\mu}\left(\vec{grad}(div(\vec{F}))+k^2\vec{F}\right)$
\end{tcolorbox}

Et le champ électrique rayonné:
\begin{tcolorbox}[colframe=red,width=4cm]
\centering
$\vec{E}=-\dfrac{1}{\epsilon}\vec{rot}(\vec{F})$
\end{tcolorbox}
\end{frame}

%%%%%%%%%%%%

\begin{frame}
\frametitle{Rayonnement en champ lointain}
De même que pour les courants électriques, en champ lointain on peut faire l'approximation:

\begin{itemize}
    \item Pour les phases: $||\vec{r}-\vec{r'}|| \approx ||\vec{r}|| - \vec{e_r}.\vec{r'}$
    \item Pour les amplitudes: $||\vec{r}-\vec{r'}|| \approx ||\vec{r}||$
\end{itemize}

\vspace{-0.25cm}
\[\Psi(\vec{r}-\vec{r'}) \approx \dfrac{e^{-jk||\vec{r}||}}{||\vec{r}||}e^{jk.\vec{e_r}.\vec{r'}}\]

\begin{tcolorbox}[colframe=red,width=9cm]
\centering
$\vec{F}(\vec{r}) \approx \dfrac{\epsilon}{4\pi}\dfrac{e^{-jk||\vec{r}||}}{||\vec{r}||}\iiint_{sources}\vec{M}(\vec{r'})e^{jk.\vec{e_r}.\vec{r'}}dV'$
\end{tcolorbox}

On peut également montrer que:

\begin{tcolorbox}[colframe=red,width=6cm]
\centering
$\vec{H}(\vec{r}) \approx j\omega\vec{e_r}\wedge(\vec{e_r}\wedge\vec{F}(\vec{r}))$
\end{tcolorbox}

\begin{tcolorbox}[colframe=red,width=5cm]
\centering
$\vec{E}(\vec{r}) \approx j.\omega.\eta(\vec{e_r}\wedge\vec{F}(\vec{r}))$
\end{tcolorbox}

\end{frame}

%%%%%%%%%%%%%

\section{Principe d'équivalence}
\begin{frame}
\frametitle{Situations``équivalentes''}

\begin{minipage}{0.45\textwidth}
    \centering\includegraphics[width=0.7\textwidth]{figures/champ_charges_elec.png}

    Champ électrique créé par 2 disques de charges électriques
\end{minipage}
\begin{minipage}{0.45\textwidth}
    \centering\includegraphics[width=0.7\textwidth]{figures/champ_charges_magn.png}

    Champ magnétique créé par 2 disques de charges
    magnétiques = champ d'un aimant
    permanent?
\end{minipage}

\end{frame}

%%%%%%%%%%%%%
\begin{frame}
    \frametitle{Conditions aux limites}
Soit $S$ une surface de discontinuité entre 2 matériaux:

\begin{figure}
    \centering\includegraphics[width=0.3\textwidth]{figures/conditions_limites.png}
\end{figure}

A la traversée de $S$, les champs sont discontinus, et l'on a:

\[\vec{J_S}=\vec{n}\wedge(\vec{H_2}-\vec{H_1})\]

\[\vec{M_S}=-\vec{n}\wedge(\vec{E_2}-\vec{E_1})\]

\end{frame}

%%%%%%%%%%%%%%
\begin{frame}
    \frametitle{Principe d'équivalence}

En l'absence de discontinuité des propriétés des matériaux, les
champs sont continus à la traversée de $S$, et les courants surfaciques
sont donc nuls.

\vspace{0.5cm}
Mais, si l'on souhaite calculer le champ du côté 2 uniquement, on
peut remplacer le problème réel par un problème fictif où les champs
seraient nuls du côté 1, mais avec l'existence de courants surfaciques,
prenant en compte la discontinuité réelle.

\begin{minipage}{0.45\textwidth}
\begin{figure}
    \centering\includegraphics[width=\textwidth]{figures/conditions_limites2.png}
\end{figure}
\end{minipage}
\begin{minipage}{0.45\textwidth}
\centering
\begin{tcolorbox}[colframe=red,width=4cm]
\centering
$\vec{J_S}=\vec{n}\wedge\vec{H_1}$

$\vec{M_S}=-\vec{n}\wedge\vec{E_1}$
\end{tcolorbox}
\end{minipage}
\end{frame}

%%%%%%%%%%%%%%
\begin{frame}
    \frametitle{Principe d'équivalence}

Dans la situation fictive équivalente, nous sommes en présence de
courants électrique et de courants magnétiques portés par la surface.

\vspace{0.5cm}
Ainsi retrouve-t-on la notion de courant magnétique, dans les
logiciels de calcul de champ: ces courants magnétiques permettent
souvent de remplacer une situation complexe réelle, par une situation
fictive, plus simple, mais qui comporte des courants magnétiques.

\end{frame}

%%%%%%%%%%%%%%
\section{Diffraction par une surface fermée}
\begin{frame}
\frametitle{Diffraction par une surface fermée}
Si $S$ est une surface fermée, le principe d'équivalence permet de
calculer le problème de diffraction, et les champs à l'extérieur de $S$
sont donnés par:

\vspace{0.5cm}
$\vec{E}(\vec{r})=\dfrac{1}{j4\pi\omega\epsilon}\left[\vec{grad}(div(\ ))+k^2\ \right]\ \iint_S \vec{J_S}(\vec{r'})\Psi(\vec{r}-\vec{r'})d^2r'-\dfrac{1}{4\pi}\vec{rot}\iint_S\vec{M_S}(\vec{r'})\Psi(\vec{r}-\vec{r'})d^2r'$

$\vec{H}(\vec{r})=\dfrac{1}{j4\pi\omega\mu}\left[\vec{grad}(div(\ ))+k^2\ \right]\ \iint_S \vec{M_S}(\vec{r'})G(\vec{r}-\vec{r'})d^2r'+\dfrac{1}{4\pi}\vec{rot}\iint_S\vec{J_S}(\vec{r'})G(\vec{r}-\vec{r'})d^2r'$

\vspace{0.5cm}
En cherchant la limite des expressions ci-dessus en champ lointain,
on obtient les formules de Kottler:

\vspace{0.25cm}
\begin{tcolorbox}[colframe=red,width=14cm]
%$\vec{E}(\vec{r})\underset{r \rightarrow \infty}{\approx}\dfrac{je^{-jkr}}{2\lambda r}\left[\eta.\iint_S\left[(\vec{n}\wedge\vec{H}(\vec{r'}))\wedge\vec{u}\right]\wedge\vec{u}e^{-jk\vec{u}.\vec{r'}}d^2r'+\iint_S\left[\vec{n}\wedge\vec{E}(\vec{r'})\right]\wedge\vec{u}e^{jk\vec{u}.\vec{r'}}d^2r'\right]$
\hspace{-0.25cm}$\vec{E}(\vec{r})\underset{r \rightarrow \infty}{\approx}\dfrac{je^{-jkr}}{2\lambda r}\left(\eta.\iint_S\vec{e_r}\wedge\left[(\vec{e_r}\wedge\vec{J_S}(\vec{r'}))\right]e^{jk\vec{e_r}.\vec{r'}}d^2r'+\iint_S\left[\vec{e_r}\wedge\vec{M}(\vec{r'})\right]e^{jk\vec{e_r}.\vec{r'}}d^2r'\right)$
\end{tcolorbox}

\vspace{0.25cm}
$\vec{H}(\vec{r})=\dfrac{1}{\eta}\vec{e_r}\wedge\vec{E}(\vec{r})$
\end{frame}

%%%%%%%%%%%%%%
\begin{frame}
\frametitle{Diffraction par une surface fermée}

\underline{Remarques:}

\vspace{0.5cm}
Lorsque la surface $S$ est un plan infini, on montre que les deux intégrales
de la page précédente sont rigoureusement égales.

\vspace{0.5cm}
On montre aussi que ces deux intégrales sont égales
pour toute surface fermée.

\begin{tcolorbox}[colframe=red,width=10cm]
    %$\vec{E}(\vec{r})\underset{r \rightarrow \infty}{\approx}\dfrac{je^{-jkr}}{2\lambda r}\left[\eta.\iint_S\left[(\vec{n}\wedge\vec{H}(\vec{r'}))\wedge\vec{u}\right]\wedge\vec{u}e^{-jk\vec{u}.\vec{r'}}d^2r'+\iint_S\left[\vec{n}\wedge\vec{E}(\vec{r'})\right]\wedge\vec{u}e^{jk\vec{u}.\vec{r'}}d^2r'\right]$
    $\vec{E}(\vec{r})\underset{r \rightarrow \infty}{\approx}\dfrac{je^{-jkr}}{2\lambda r}\left(\eta.\iint_S\vec{e_r}\wedge\left[(\vec{e_r}\wedge 2\vec{J_S}(\vec{r'}))\right]e^{jk\vec{e_r}.\vec{r'}}d^2r'\right)$
    
    \vspace{0.25cm}
    ou

    \vspace{0.25cm}
    $\vec{E}(\vec{r})\underset{r \rightarrow \infty}{\approx}\dfrac{je^{-jkr}}{2\lambda r}\left(\iint_S\left[\vec{e_r}\wedge 2\vec{M}(\vec{r'})\right]e^{jk\vec{e_r}.\vec{r'}}d^2r'\right)$
    \end{tcolorbox}

\end{frame}

%%%%%%%%%%%%%%
\section{Doublet magnétique}
\begin{frame}
    \frametitle{Doublet magnétique}
On étudie le rayonnement d'une ``petite'' fente de surface $S$
(petite devant la longueur d'onde) sur laquelle on applique un
champ électrique tangentiel:
    
\begin{figure}
    \centering\includegraphics[width=0.5\textwidth]{figures/doublet_magnetique.png}
\end{figure}

\begin{center}
    Avec $M=E$
\end{center}

\end{frame}
    
%%%%%%%%%%%%%%
\begin{frame}
    \frametitle{Rayonnement du doublet magnétique}
On obtient (pour une fente ``verticale'', portée par Oz):

\begin{center}
\begin{tcolorbox}[colframe=red,width=7cm]
    $\vec{H}(\vec{r})=\dfrac{j.M.S}{2.\eta.\lambda}\dfrac{e^{-jkr}}{r}\sin\theta.\vec{u_\theta}$
\end{tcolorbox}
\end{center}

Le rayonnement d'un doublet magnétique est donc obtenu à partir du
rayonnement du doublet électrique en remplaçant les lignes de champs
du champ électrique par celles du champ magnétique. Et plus
précisément:

\begin{center}
    \begin{tcolorbox}[colframe=red,width=3cm]
\centering $I\ l \rightarrow M\ S$

$E \rightarrow \eta \ H$
\end{tcolorbox}
\end{center}

\end{frame}
    
%%%%%%%%%%%%%%
\begin{frame}
    \frametitle{Rayonnement du doublet magnétique}
Les lignes de champ de E et H sont échangées, mais la puissance
émise (par unité d'angle solide) varie de la même façon en fonction
des directions:

\begin{minipage}{0.45\textwidth}
    \centering\includegraphics[width=0.7\textwidth]{figures/rayonnement_doublet_magn_2D.png}
\end{minipage}
\begin{minipage}{0.45\textwidth}
    \centering\includegraphics[width=0.7\textwidth]{figures/rayonnement_doublet_magn_3D.png}
\end{minipage}

\vspace{0.5cm}
\begin{center}
$r(\theta,\varphi)=\sin^2\theta$
\hspace{2cm}
Directivité: $D=\dfrac{3}{2}$
\end{center}
\end{frame}
    
%%%%%%%%%%%%%%
\section{Ouverture rayonnante plane}
\begin{frame}
    \frametitle{Ouverture rayonnante plane}
\begin{figure}
    \centering\includegraphics[width=0.25\textwidth]{figures/antenne_cornet.png}
\end{figure}

On peut calculer les propriétés de rayonnement d'une ouverture
plane par la seule connaissance de la valeur du champ sur celle-ci:

\begin{minipage}{0.45\textwidth}
    \centering\includegraphics[width=0.7\textwidth]{figures/champ_ouverture.png}
\end{minipage}
\begin{minipage}{0.45\textwidth}
    \centering\includegraphics[width=0.7\textwidth]{figures/diagramme_ouverture.png}
\end{minipage}
    
\end{frame}

%%%%%%%%%%%%%%
\begin{frame}
\frametitle{Ouverture rayonnante plane}

\begin{figure}
\centering\includegraphics[width=0.5\textwidth]{figures/principe_equivalence_ouverture.png}
\end{figure}


\end{frame}

%%%%%%%%%%%%%%
\begin{frame}
    \frametitle{Rayonnement à l'infini d'une ouverture plane}

A l'infini, le champ électrique d'une ouverture plane dans le plan $(\vec{x};\vec{y})$ est donné par:

$\vec{E}(\vec{r})\underset{r \rightarrow \infty}{\approx}\dfrac{je^{-jkr}}{2\lambda r}\left(\iint_S\left[\vec{e_r}\wedge 2\vec{M}(\vec{r'})\right]e^{jk\vec{e_r}.\vec{r'}}d^2r'\right)$

\begin{tcolorbox}[colframe=red,width=13cm]
    $\vec{E}(\vec{r})\underset{r \rightarrow \infty}{\approx}\dfrac{je^{-jkr}}{\lambda r}\left(\iint_S\left[\vec{e_r}\wedge (\vec{E}(x',y',0)\wedge\vec{e_z})\right]e^{jk(\sin\theta\cos\varphi x'+ \sin\theta\sin\varphi y')}dx'dy'\right)$
    %$\vec{E}(r\ \vec{u_0}) \underset{r \rightarrow \infty}{\approx} \dfrac{je^{-jkr}}{\lambda r}u_{z0}\iint_S\vec{E}(x',y',0) e^{+jk(x'.u_{x0}+y'.u_{y0})}dx'dy'$
\end{tcolorbox}

\begin{figure}
    \centering\includegraphics[width=0.3\textwidth]{figures/champ_ouverture2.png}
\end{figure}

\end{frame}
    
%%%%%%%%%%%%%%
% \begin{frame}
% \frametitle{Directivité d'une ouverture plane}

% $U(u_x,u_y)=\dfrac{r^2||\vec{E}||^2}{2\eta}=\dfrac{1}{2\eta\lambda^2}u_z^2\left|\left|\iint_S\vec{E}(x',y',0)e^{jk(u_x\ x'+u_y\ y')}dx'dy'\right|\right|^2$

% \vspace{0.25cm}
% $U(u_x,u_y)$ est maximum pour une certaine direction $(u_{x0},u_{y0})$ de l'espace.

% \vspace{0.25cm}
% $U_{\max}=\dfrac{r^2||\vec{E}||^2}{2\eta}=\dfrac{1}{2\eta\lambda^2}u_{z0}^2\left|\left|\iint_S\vec{E}(x',y',0)e^{jk(u_{x0}\ x'+u_{y0}\ y')}dx'dy'\right|\right|^2$

% \vspace{0.25cm}
% $P_{ray}=\dfrac{1}{2\eta}\iint_S\left|\left|E_{xy}\right|\right|^2dx'dy'$

% \vspace{0.5cm}
% Donc

% \begin{tcolorbox}[colframe=black,width=11cm]
%     $D=4\pi\dfrac{U_{\max}}{P_{ray}}=\dfrac{4\pi}{\lambda^2}\dfrac{u_{z0}^2\left|\left|\iint_S\vec{E}(x',y',0)e^{jk(u_{x0}\ x'+u_{y0}\ y')}dx'dy'\right|\right|^2}{\iint_S\left|\left|\vec{E}_{xy}\right|\right|^2dx'dy'}$
% \end{tcolorbox}

% \end{frame}

%%%%%%%%%%%%%%
\begin{frame}
\frametitle{Directivité d'une ouverture plane}
%Or $u_{z0}^2\left|\left|\iint_S\vec{E}(x',y',0)e^{jk(u_{x0}\ x'+u_{y0}\ y')}dx'dy'\right|\right|^2 \leq S\iint_S\left|\left|E_{xy}\right|\right|^2dx'dy'$

\vspace{0.5cm}
%Donc la directivité présente un maximum absolu:
La directivité d'une ouverture plane de surface $S$ présente un maximum absolu:

\begin{center}
    \begin{tcolorbox}[colframe=red,width=4cm]
        $D_{\max}=\dfrac{4\pi S}{\lambda^2}$
    \end{tcolorbox}
\end{center}

\vspace{0.5cm}
L'égalité peut être obtenue pour une ouverture dont le champ électrique $\vec{E}(x',y',0)e^{jk(\sin\theta\cos\varphi x'+\sin\theta\sin\varphi y')}$ est constant sur toute la surface de l'ouverture.

\end{frame}

%%%%%%%%%%%%%%
\begin{frame}
\frametitle{\vspace{-0.5cm}maximum sur la normale\\
\vspace{-0.5cm}Cas d'une ouverture à polarisation linéaire, avec un rayonnement}

En prenant, par exemple, une polarisation parallèle à Oy:

\begin{center}
    \begin{tcolorbox}[colframe=red,width=6cm]
        $D_{\max}=k_{apperture}\dfrac{4\pi S}{\lambda^2}$
    \end{tcolorbox}
\end{center}

Où le \textbf{coefficient d'efficacité d'ouverture} $k_{apperture}$
est donné par:

\begin{tcolorbox}[colframe=red,width=11cm]
    $k_{apperture}=\dfrac{S{efficace}}{S_{\text{géométrique}}}=\dfrac{\left|\left|\iint_S E_y dx'dy'\right|\right|^2}{S_{\text{géométrique}}\iint_S\left|\left|E_{y}\right|\right|^2dx'dy'}\leq 1$
\end{tcolorbox}

\vspace{0.5cm}
Une valeur $50\% \leq k \leq 80\%$ est assez courante.

\end{frame}

%%%%%%%%%%%%%%
\section{Cornet rectangulaire et circulaire}
\begin{frame}
\frametitle{Cornet rectangulaire}
\begin{minipage}{0.45\textwidth}
    \centering\includegraphics[width=0.7\textwidth]{figures/champ_cornet.png}
\end{minipage}
\begin{minipage}{0.45\textwidth}
    \centering\includegraphics[width=0.7\textwidth]{figures/champ_cornet2.png}
\end{minipage}

\centering
\vspace{-1cm}
\begin{minipage}{0.45\textwidth}
    \centering\includegraphics[width=0.6\textwidth]{figures/diagramme_cornet.png}
\end{minipage}

\end{frame}
%%%%%%%%%%%%%%
\begin{frame}
\frametitle{Cornet rectangulaire}

\begin{minipage}{0.45\textwidth}
    \centering\includegraphics[width=0.7\textwidth]{figures/diagramme_cornet_3D.png}
\end{minipage}
\begin{minipage}{0.45\textwidth}
    \centering\includegraphics[width=0.7\textwidth]{figures/diagramme_cornet_2D.png}
\end{minipage}

Surface de l'ouverture du cornet: 67,56 mm x 49,53 mm

Valeur de l'efficacité de l'ouverture?

\end{frame}

%%%%%%%%%%%%%%
\begin{frame}
\frametitle{Cornet rectangulaire: modèle approché}
On modélise un cornet de dimensions $2a \times 2b$ par une ouverture sur
laquelle existe un champ du type ``TE$_{10}$ dilaté'', avec une phase
sphérique:

\begin{figure}
    \centering\includegraphics[width=0.7\textwidth]{figures/modele_cornet.png}
\end{figure}

$E_y(x,y)=E_0\cos\left(\dfrac{\pi x}{2a}\right)e^{-j\dfrac{2\pi}{\lambda}\sqrt{x^2+y^2+d^2}}$


\end{frame}

%%%%%%%%%%%%%%
\begin{frame}
\frametitle{Cornet rectangulaire: modèle approché}

Soit, pour un cornet tel que 2a = 5$\lambda$, 2b = 2.5$\lambda$, d = 5$\lambda$:

\begin{figure}
    \centering\includegraphics[width=0.7\textwidth]{figures/champ_cornet3.png}
\end{figure}

\end{frame}

%%%%%%%%%%%%%%
\begin{frame}
\frametitle{Cornet rectangulaire: modèle approché}
Diagramme de rayonnement:

\begin{figure}
    \centering\includegraphics[width=0.7\textwidth]{figures/diagramme_cornet_3D2.png}
\end{figure}

\end{frame}

%%%%%%%%%%%%%%
\begin{frame}
\frametitle{Cornet rectangulaire: modèle approché}
Diagramme de rayonnement:

\begin{minipage}{0.45\textwidth}
    \centering\includegraphics[width=\textwidth]{figures/diagramme_cornet_2D2.png}
\end{minipage}
\begin{minipage}{0.45\textwidth}
    \centering\includegraphics[width=\textwidth]{figures/diagramme_cornet_2D3.png}
\end{minipage}


\end{frame}

%%%%%%%%%%%%%%
\begin{frame}
\frametitle{Intérêt d'un cornet, par rapport à
un simple guide d'onde ouvert}

Plus grande surface d'ouverture $\Rightarrow$ antenne plus directive

\vspace{1cm}
Coefficient de réflexion quasi nul dans une ``large'' bande $\Rightarrow$
meilleure adaptation

\end{frame}

%%%%%%%%%%%%%%
\begin{frame}
\frametitle{Cornet circulaire: modèle approché}
Mode TE$_{11}$:

\begin{minipage}{0.35\textwidth}
    \centering\includegraphics[width=\textwidth]{figures/cornet_circulaire.png}
\end{minipage}
\begin{minipage}{0.55\textwidth}
    \centering\includegraphics[width=\textwidth]{figures/champ_cornet_circulaire.png}
\end{minipage}

\end{frame}

%%%%%%%%%%%%%%
\begin{frame}
\frametitle{Cornet circulaire: modèle approché}

Composante verticale du champ sur l'ouverture:
\begin{figure}
\centering\includegraphics[width=0.8\textwidth]{figures/champ_cornet_circulaire2.png}
\end{figure}

\end{frame}

%%%%%%%%%%%%%%
\begin{frame}
\frametitle{Cornet circulaire: modèle approché}
Diagramme en polarisation verticale:

\begin{figure}
    \centering\includegraphics[width=0.8\textwidth]{figures/diagramme_cornet_circulaire.png}
\end{figure}

\end{frame}

%%%%%%%%%%%%%%
\begin{frame}
\frametitle{Cornet circulaire: modèle approché}
\begin{minipage}{0.45\textwidth}
    \centering\includegraphics[width=\textwidth]{figures/diagramme_cornet_circulaire_2D.png}
\end{minipage}
\begin{minipage}{0.45\textwidth}
    \centering\includegraphics[width=\textwidth]{figures/diagramme_cornet_circulaire_2D2.png}
\end{minipage}


\end{frame}

%%%%%%%%%%%%%%
\begin{frame}
\frametitle{Cornet circulaire: modèle approché}

Composante horizontale du champ sur l'ouverture:

\begin{figure}
    \centering\includegraphics[width=0.8\textwidth]{figures/champ_cornet_circulaire_horizontal.png}
\end{figure}

\end{frame}

%%%%%%%%%%%%%%
\begin{frame}
\frametitle{Cornet circulaire: modèle approché}
Diagramme de rayonnement en polarisation croisée:

\begin{figure}
    \centering\includegraphics[width=0.8\textwidth]{figures/diagramme_cornet_circulaire_croise.png}
\end{figure}

\end{frame}

%%%%%%%%%%%%%%
\section{Antenne à réflecteur parabolique}
\begin{frame}
\frametitle{Antenne à réflecteur parabolique}

\begin{minipage}{0.35\textwidth}
    \centering\includegraphics[width=\textwidth]{figures/parabole.png}
\end{minipage}
\begin{minipage}{0.55\textwidth}
    \centering Ouverture équiphase:
    
    \includegraphics[width=0.5\textwidth]{figures/equiphase_parabole.png}

    FHM = constante
\end{minipage}

\end{frame}

%%%%%%%%%%%%%%
\begin{frame}
\frametitle{Antenne à réflecteur parabolique}

Ouverture équi-phase $\Rightarrow$ Gain et directivité maximum possible
pour la surface choisie.

\vspace{1cm}
On obtient souvent une valeur de k proche de 0,8.

\end{frame}

\end{document}