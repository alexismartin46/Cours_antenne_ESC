\documentclass[
10pt,
aspectratio=169,
]{beamer}


\usepackage{lipsum}
\usepackage{tikz}
\usepackage{bm}
\usepackage{tcolorbox}
\usepackage{beamerthemeensea}


\title{Antennes – ESC}
\subtitle{Cours 6: Antennes Réseaux}
\date{\the\year}
\author{Alexis MARTIN}
%\institute{ENSEA}

\usetheme{ensea}  


\begin{document}

\begin{frame}
\titlepage\end{frame}

\begin{frame}
\tableofcontents
\end{frame}

%%%%%%%%%%%%

\begin{frame} 
\frametitle{Antennes Réseaux}

\begin{minipage}{0.3\textwidth}
\centering\includegraphics[width=0.9\textwidth]{figures/reseau_patchs.png}
\end{minipage}
\begin{minipage}{0.3\textwidth}
\centering\includegraphics[width=\textwidth]{figures/reseau_fentes.png}
\end{minipage}
\begin{minipage}{0.3\textwidth}
\centering\includegraphics[width=0.7\textwidth]{figures/reseau_yagi.png}
\end{minipage}

\begin{minipage}{0.3\textwidth}
\centering\includegraphics[width=0.9\textwidth]{figures/reseau_filaire.png}
\end{minipage}
\begin{minipage}{0.3\textwidth}
\centering\includegraphics[width=\textwidth]{figures/reseau_radome.png}
\end{minipage}
\begin{minipage}{0.3\textwidth}
\centering\includegraphics[width=\textwidth]{figures/reseau_paraboles.png}
\end{minipage}

\end{frame}

%%%%%%%%%%%%
\section{Facteur de réseau}
\begin{frame}
\frametitle{Translation d'une antenne}

Soit une antenne placée à l'origine, avec une répartition de courant
connue. Le champ rayonné à l'infini est donné par:

\vspace{0.5cm}
$\vec{E}(\vec{r})\underset{r \rightarrow \infty}{\approx}\dfrac{j.\omega.\mu}{4\pi}.\dfrac{e^{-j.k.r}}{r}\vec{u}\wedge \left(\vec{u}\wedge \left[\iiint_{W_S}\vec{J}(\vec{r'})e^{jk\vec{u}.\vec{r'}}d^3r'\right]\right)$

\begin{minipage}{0.45\textwidth}
On translate l'antenne au point $r_1$:
\end{minipage}
\begin{minipage}{0.45\textwidth}
\centering\includegraphics[width=\textwidth]{figures/translation_antenne.png}
\end{minipage}

\end{frame}

%%%%%%%%%%%%
\begin{frame}
\frametitle{Translation d'une antenne}

Pour un point d'observation éloigné, on a:

\[||\vec{r}-(\vec{r_1}+\vec{r'})||\underset{r \rightarrow \infty}{\approx} r-(\vec{r_1}+\vec{r'}).\vec{u}\]

L'expression du champ rayonné par l'antenne translatée est alors:

\vspace{0.5cm}
\begin{tcolorbox}[colframe=red,width=5cm]
\centering$\vec{E_1}(\vec{r})=\vec{E_0}(\vec{r})e^{jk\vec{u}.\vec{r_1}}$
\end{tcolorbox}

\vspace{0.5cm}
\begin{tcolorbox}[colframe=red,width=9cm]
\centering Ce résultat très simple, et très important est appelé \textit{Théorème de
translation}
\end{tcolorbox}


\end{frame}

%%%%%%%%%%%%
\begin{frame}
\frametitle{Facteur de réseau}

\begin{minipage}{0.45\textwidth}
Soit $\vec{E_0}(\vec{r})$ le champ rayonné par une antenne
élémentaire placée à l'origine, et parcourue par le
courant $I_0$, et soit $I_p$ le courant sur l'antenne $p$,
translatée de l'antenne placée à l'origine, on a alors:

\[\vec{E}(\vec{r})=\vec{E_0}(\vec{r})\sum_{p=0}^{N-1}\dfrac{I_p}{I_0}e^{j.p.k.d.\vec{u}.\vec{e_z}}\]

\end{minipage}
\begin{minipage}{0.45\textwidth}
\centering\includegraphics[width=0.8\textwidth]{figures/facteur_reseau.png}
\end{minipage}

\vspace{0.25cm}
$f(\vec{u})=\sum_{p=0}^{N-1}\dfrac{I_p}{I_0}e^{j.p.k.d.\vec{u}.\vec{e_z}}$ est appelé ``\textit{facteur de réseau}''.

\vspace{0.5cm}
\underline{\textbf{Attention}}: $I_p$ et $I_0$ sont des courants complexes, il faut donc tenir compte de leurs amplitudes et de leurs phases!
\end{frame}

%%%%%%%%%%%%
\begin{frame}
\frametitle{Diagramme de rayonnement}

Soit $r_0(\theta,\varphi)$ la caractéristique de rayonnement de l'antenne
élémentaire. On a:

$r(\theta,\varphi)=r_0(\theta,\varphi).|f(\theta,\varphi)|^2$

\vspace{0.5cm}
Le diagramme de rayonnement d'un réseau est donc obtenu en
multipliant le diagramme de rayonnement de l'antenne élémentaire
par $|f(\theta,\varphi)|^2$.

\vspace{0.5cm}
Il y a un effet de ``\textit{multiplication des diagrammes}''.

\end{frame}

%%%%%%%%%%%%
\section{Réseau uniforme à déphasage linéaire}
\begin{frame}
\frametitle{Réseau uniforme à déphasage linéaire}

Le courant a la même amplitude sur toutes les antennes, mais la phase
varie linéairement:

$I_p=I_0.e^{-j.p.\Psi}$

On a alors: $f(\theta)=e^{j\frac{N-1}{2}(k.d.\sin\theta-\Psi)}\dfrac{\sin\left(\dfrac{N(k.d.\sin\theta-\Psi)}{2}\right)}{\sin\left(\dfrac{k.d.\sin\theta-\Psi}{2}\right)}$

Et, \textbf{après normalisation, de façon à ce que le réseau rayonne la
même puissance totale que l'antenne élémentaire}:

\[||f(\theta)||=\dfrac{1}{\sqrt{N}}\left|\dfrac{\sin\left(\dfrac{N(k.d.\sin\theta-\Psi)}{2}\right)}{\sin\left(\dfrac{k.d.\sin\theta-\Psi}{2}\right)}\right|\]

\end{frame}

%%%%%%%%%%%%
\begin{frame}
\frametitle{Réseau uniforme à déphasage linéaire}

\vspace{-0.25cm}
\begin{figure}
\centering
\includegraphics[width=0.7\textwidth]{figures/reseau_lineaire_psi0.png}
\end{figure}

\vspace{-0.4cm}
\begin{center}
Réseau uniforme à déphasage linéaire, $\Psi=0$
\end{center}

\vspace{-0.25cm}
\[||f(u_z)||=\dfrac{1}{\sqrt{N}}\left|\dfrac{\sin\left(\dfrac{N(k.d.\vec{u}.\vec{e_z}-\Psi)}{2}\right)}{\sin\left(\dfrac{k.d.\vec{u}.\vec{e_z}-\Psi}{2}\right)}\right|\]

\end{frame}

%%%%%%%%%%%%
\begin{frame}
\frametitle{Réseau uniforme à déphasage linéaire}

\vspace{-0.25cm}
\begin{figure}
\centering
\includegraphics[width=0.7\textwidth]{figures/reseau_lineaire_psi_non0.png}
\end{figure}

\vspace{-0.4cm}
\begin{center}
Réseau uniforme à déphasage linéaire, $\Psi\ne 0$
\end{center}

\vspace{-0.25cm}
\[||f(u_z)||=\dfrac{1}{\sqrt{N}}\dfrac{\sin\left(\dfrac{N(kdu_z-\Psi)}{2}\right)}{\sin\left(\dfrac{kdu_z-\Psi}{2}\right)}\]

\end{frame}

%%%%%%%%%%%%
\section{Exemples}
\begin{frame}
\frametitle{Exemple $N=8$, $d = \lambda/2$}

\begin{minipage}{0.55\textwidth}
\centering\includegraphics[width=\textwidth]{figures/exemple_N8_dls2_psi0_rect.png}
\end{minipage}
\begin{minipage}{0.35\textwidth}
\centering\includegraphics[width=\textwidth]{figures/exemple_N8_dls2_psi0_polar.png}
\end{minipage}

\begin{center}
Réseau uniforme à déphasage linéaire, $\Psi=0$
\end{center}

Ce réseau est dit ``\textit{transverse}'' (broadside). Il rayonne
perpendiculairement à la direction de l'alignement.

\end{frame}

%%%%%%%%%%%%
\begin{frame}
\frametitle{Exemple $N=8$, $d = \lambda/2$}

\begin{minipage}{0.55\textwidth}
\centering\includegraphics[width=\textwidth]{figures/exemple_N8_dls2_psipisqrt2_rect.png}
\end{minipage}
\begin{minipage}{0.35\textwidth}
\centering\includegraphics[width=\textwidth]{figures/exemple_N8_dls2_psipisqrt2_polar.png}
\end{minipage}

\begin{center}
Réseau uniforme à déphasage linéaire, $\Psi=\dfrac{\pi}{\sqrt{2}}$
\end{center}

Ce réseau rayonne un maximum à $45^\circ$ de la direction de l'alignement.

\end{frame}

%%%%%%%%%%%%
\begin{frame}
\frametitle{Exemple $N=8$, $d = \lambda/2$}

\begin{minipage}{0.55\textwidth}
\centering\includegraphics[width=\textwidth]{figures/exemple_N8_dls2_psipi_rect.png}
\end{minipage}
\begin{minipage}{0.35\textwidth}
\centering\includegraphics[width=\textwidth]{figures/exemple_N8_dls2_psipi_polar.png}
\end{minipage}

\begin{center}
Réseau uniforme à déphasage linéaire, $\Psi=\pi$
\end{center}

Ce réseau est dit ``\textit{longitudinal}'' (end-fire). Il rayonne parallèlement
à la direction de l'alignement.
\end{frame}

%%%%%%%%%%%%
\begin{frame}
\frametitle{Lobes de réseau}

Des lobes de réseau apparaissent (dans le domaine du visible) si $d>\lambda$

\begin{figure}
\centering
\includegraphics[width=0.9\textwidth]{figures/lobes_reseau.png}
\end{figure}

\end{frame}

%%%%%%%%%%%%
\begin{frame}
\frametitle{Réseau transverse de doublets
électriques}

\vspace{-0.25cm}
\begin{figure}
\centering
\includegraphics[width=0.65\textwidth]{figures/directivite_reseau_doublet.png}
\end{figure}

La directivité d'un doublet élémentaire est égale à 1.5.

Pour obtenir une directivité égale à 15, avec N = 10, il faut choisir un
pas de l'ordre de $0.75 \lambda$.

\end{frame}

%%%%%%%%%%%%
\begin{frame}
\frametitle{Réseau transverse de dipôles
demi-onde}

\vspace{-0.25cm}
\begin{figure}
\centering
\includegraphics[width=0.65\textwidth]{figures/directivite_reseau_dipole.png}
\end{figure}

La directivité d'un dipôle élémentaire est égale à 1.64.

Pour obtenir une directivité égale à 16.4, avec N = 10, il faut choisir un
pas légèrement suppérieur à $0.8 \lambda$.

\end{frame}

%%%%%%%%%%%%
\section{Alimentation d'un réseau}
\begin{frame}
\frametitle{Alimentation d'un réseau}

En émission, le signal issu de la source doit être divisé, avant
d'alimenter chaque élément du réseau.

Et si l'antenne est destinée à faire du ``\textit{balayage de phase}'', le
signal doit aussi être déphasé avant chaque élément.

\vspace{1cm}
En réception, le répartiteur se comporte en additionneur:

\begin{figure}
\centering
\includegraphics[width=0.7\textwidth]{figures/reseau_alimentation.png}
\end{figure}

\end{frame}

%%%%%%%%%%%%%
\begin{frame}
\frametitle{Conception du répartiteur}

La conception du répartiteur est complexe car les antennes constituant
le réseau sont fortement couplées entre elles:

\begin{figure}
\centering
\includegraphics[width=0.7\textwidth]{figures/reseau_couplage.png}
\end{figure}

Les conséquences du couplage sont doubles:
\begin{itemize}
    \item l'impédance de chaque élément dépend non seulement de la
    présence des autres éléments, mais aussi de la nature exacte du
    coupleur
    \item le diagramme de rayonnement n'est pas le même que celui
    que l'on aurait avec des antennes découplées entre elles. Il dépend
    de la nature exacte du coupleur
\end{itemize}

\end{frame}

%%%%%%%%%%%%%
\begin{frame}
\frametitle{Coefficient de réflexion ``actif''}

Il s'agit du coefficient de réflexion d'un élément de l'antenne, alors
que tous les éléments sont connectés au répartiteur:

\begin{figure}
\centering
\includegraphics[width=0.7\textwidth]{figures/reseau_reflexion.png}
\end{figure}

Pour l'antenne 1, ce coefficient est donné par:

\[\Gamma_{1\ \text{actif}}=\dfrac{b_1}{a_1}\]

\end{frame}

%%%%%%%%%%%%%
\begin{frame}
\frametitle{Cas d'un réseau infini}

On effectue la simulation sur une seule cellule d'un réseau infini:

\begin{figure}
\centering
\includegraphics[width=0.5\textwidth]{figures/reseau_infini.png}
\end{figure}

On obtient alors le coefficient de réflexion (actif) de la cellule
élémentaire, avec un choix quelconque de la direction d'émission.

Le coefficient de réflexion obtenu dépend de la direction souhaitée.

\vspace{0.5cm}
La détermination du coefficient de réflexion actif facilite la
conception du répartiteur pour un réseau infini.

\end{frame}

%%%%%%%%%%%%%
\begin{frame}
\frametitle{Angle d'aveuglement}

On observe une direction pour laquelle toute la puissance est réfléchie:

Il s'agit de ``\textit{l'angle d'aveuglement}'' (``\textit{scan blindness angle}'').

\vspace{0.5cm}
Cet angle peut aussi être observé, à l'aide de la courbe de gain, en
fonction de l'angle ``visé''.

\begin{minipage}{0.45\textwidth}
\centering\includegraphics[width=\textwidth]{figures/angle_aveuglement_s11.png}
\end{minipage}
\begin{minipage}{0.45\textwidth}
\centering\includegraphics[width=\textwidth]{figures/angle_aveuglement_gain.png}
\end{minipage}

\end{frame}

%%%%%%%%%%%%%
\begin{frame}
\frametitle{Répartiteur parfaitement découplé}

Dans ce cas, le répartiteur présente une matrice $[S]$, qui est la
transconjuguée de la matrice $[S_A]$ des antennes.

\begin{figure}
\centering
\includegraphics[width=0.7\textwidth]{figures/repartiteur_decouple.png}
\end{figure}

C'est le cas optimum qui permet un transfert maximum de puissance
vers le réseau.

\vspace{0.5cm}
Mais, pour obtenir ce résultat, la structure du répartiteur est
particulièrement complexe!
\end{frame}

\end{document}