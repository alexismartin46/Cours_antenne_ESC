\documentclass[
10pt,
aspectratio=169,
]{beamer}


\usepackage{lipsum}
\usepackage{tikz}
\usepackage{bm}
\usepackage{tcolorbox}
\usepackage{beamerthemeensea}


\title{Antennes – ESC}
\subtitle{Cours 5: Antennes Patchs imprimés}
\date{\the\year}
\author{Alexis MARTIN}
%\institute{ENSEA}

\usetheme{ensea}  


\begin{document}

\begin{frame}
\titlepage\end{frame}

\begin{frame}
\tableofcontents
\end{frame}

%%%%%%%%%%%%

\begin{frame} 
\frametitle{Antennes Patchs}

\begin{minipage}{0.3\textwidth}
\centering\includegraphics[width=\textwidth]{figures/Modele_Patch.png}
\end{minipage}
\begin{minipage}{0.3\textwidth}
\centering\includegraphics[width=0.8\textwidth]{figures/Patch.png}
\end{minipage}
\begin{minipage}{0.3\textwidth}
\centering\includegraphics[width=0.9\textwidth]{figures/Patch2.png}
\end{minipage}

\begin{minipage}{0.3\textwidth}
\centering\includegraphics[width=0.8\textwidth]{figures/Patch3.png}
\end{minipage}
\begin{minipage}{0.3\textwidth}
\centering\includegraphics[width=\textwidth]{figures/Patch4.png}
\end{minipage}
\begin{minipage}{0.3\textwidth}
\centering\includegraphics[width=\textwidth]{figures/Patch5.png}
\end{minipage}

\end{frame}

%%%%%%%%%%%%
\section{Modélisation}
\begin{frame}
\frametitle{Modélisation approchée}

Deux méthodes d'alimentation possibles:

\begin{minipage}{0.45\textwidth}
\centering\includegraphics[width=\textwidth]{figures/alim_patch.png}
\end{minipage}
\begin{minipage}{0.45\textwidth}
\centering\includegraphics[width=\textwidth]{figures/alim_patch2.png}
\end{minipage}

Approximation de la ligne sans pertes:

\begin{minipage}{0.4\textwidth}
\centering\includegraphics[width=\textwidth]{figures/modele_patch2.png}
\end{minipage}
\begin{minipage}{0.1\textwidth}
\centering$\Leftrightarrow$
\end{minipage}
\begin{minipage}{0.4\textwidth}
\centering\includegraphics[width=\textwidth]{figures/modele_patch3.png}
\end{minipage}

\end{frame}

%%%%%%%%%%%%
\begin{frame}
\frametitle{Fréquence de résonance}
\begin{minipage}{0.45\textwidth}

Ondes stationnaires d'amplitude $V_{\max}$:

\vspace{0.5mm}
$V_{\max}=\dfrac{V_g}{\sqrt{\cos^2\left(\dfrac{\omega .l}{v}\right)+\dfrac{R_g^2}{R_c^2}\sin^2\left(\dfrac{\omega .l}{v}\right)}}$

\vspace{0.5mm}
Le cas des antennes ``patch'' correspond à $R_g >> R_c$

\vspace{0.5mm}
Les fréquences de résonance sont donc données par:

\begin{tcolorbox}[colframe=red,width=6cm]
\centering$l=n\dfrac{\lambda_g}{2}$

\vspace{0.5mm}
$\lambda_g$: longueur d'onde ``guidée''

\end{tcolorbox}

\end{minipage}
\begin{minipage}{0.45\textwidth}
\centering\includegraphics[width=0.9\textwidth]{figures/freq_res.png}
\end{minipage}

\end{frame}

%%%%%%%%%%%%
\begin{frame}
\frametitle{Courant à la résonance sur le patch $\lambda_g/2$}

$\lambda_g=\dfrac{\lambda_0}{\sqrt{\epsilon_{eff}}}$

\vspace{0.5mm}
$\lambda_0$: longueur d'onde dans le vide

\vspace{0.5mm}
$\epsilon_{eff}$: constante diélectrique ``effective'', qui dépend de la géométrie
de la ligne

\vspace{0.5mm}
Dans l'approximation de la ligne sans pertes, le courant est donné par:

\begin{figure}
\centering
\includegraphics[width=0.5\textwidth]{figures/courant_patch.png}
\end{figure}
\end{frame}

%%%%%%%%%%%%
\begin{frame}
\frametitle{Tension à la résonance sur le patch $\lambda_g/2$}
Dans l'approximation de la ligne sans pertes, la tension est donnée par:

\begin{figure}
\centering
\includegraphics[width=0.5\textwidth]{figures/tension_patch.png}
\end{figure}

\end{frame}

%%%%%%%%%%%%%
\begin{frame}\label{sec:champ_patch}
\frametitle{Champ électrique autour du patch}

\begin{minipage}{0.45\textwidth}
    Vue de ``côté'':
\end{minipage}
\begin{minipage}{0.45\textwidth}
    \centering\includegraphics[width=\textwidth]{figures/champ_patch_cote.png}
\end{minipage}

\begin{minipage}{0.45\textwidth}
    Vue de dessus:
\end{minipage}
\begin{minipage}{0.45\textwidth}
    \centering\includegraphics[width=0.7\textwidth]{figures/champ_patch_dessus.png}
\end{minipage}

\begin{minipage}{0.45\textwidth}
    Equivalence avec le rayonnement de 2 fentes:
\end{minipage}
\begin{minipage}{0.45\textwidth}
    \centering\includegraphics[width=0.9\textwidth]{figures/champ_patch_equivalent.png}
\end{minipage}

\end{frame}

%%%%%%%%%%%%%
\begin{frame}
\frametitle{Longeur effective du patch}
En pratique, à cause du débordement des lignes de champ, la longeur effective $L_e$ du patch est:

\[L_e=L+2\Delta L\]

On utilise classiquement la formule empirique:
\[\Delta L=0.412h\dfrac{(\epsilon_{eff}+0.3)(\dfrac{W}{h}+0.264)}{(\epsilon_{eff}-0.258)(\dfrac{W}{h}+0.8)}\]
\begin{itemize}
    \item $h$ la hauteur du substrat
    \item $W$ la largeur du patch
    \item $\epsilon_{eff}$ la constante diélectrique effective\\
    $\epsilon_{eff}=\dfrac{\epsilon_r+1}{2}+\dfrac{\epsilon_r-1}{2}\dfrac{1}{\sqrt{1+12\dfrac{h}{W}}}$
\end{itemize}

La résonnance se produit donc pour:
\[L_e=\dfrac{\lambda_g}{2}\]

\end{frame}

%%%%%%%%%%%%%
\begin{frame}
\frametitle{Diagramme de rayonnement}
Selon l'approximation de ligne sans pertes (\textbf{attention au repère utilisé}, Cf slide\ \ref{sec:champ_patch}):

\vspace{0.5cm}
%$||\vec{E}(u_x,u_y)||=K\sqrt{1-u_x^2}\dfrac{\sin\left(\pi u_y \dfrac{w}{\lambda}\right)}{\pi u_y \dfrac{w}{\lambda}}\cos\left(\pi u_x \dfrac{l}{\lambda}\right)$
$E_\varphi \approx K \sin\theta\dfrac{\sin\left(\dfrac{k.h\sin\theta\cos\varphi}{2}\right)}{\dfrac{k.h\sin\theta\cos\varphi}{2}}\dfrac{\sin\left(\dfrac{k.W\cos\theta}{2}\right)}{\dfrac{k.W\cos\theta}{2}}\cos\left(\dfrac{k.L_e}{2}\sin\theta\sin\varphi\right)$

\vspace{0.25cm}
avec:
\begin{itemize}
    \item $h$ la hauteur du substrat
    \item $W$ la largeur du patch
    \item $L_e$ la longueur effective du patch
\end{itemize}

\vspace{0.5cm}
Pour des substrats de faible épaisseur ($kh<<1$), on a:

\vspace{0.25cm}
$E_\varphi \approx E_0 \sin\theta\dfrac{\sin\left(\dfrac{k.W\cos\theta}{2}\right)}{\cos\theta}\cos\left(\dfrac{k.L_e}{2}\sin\theta\sin\varphi\right)$

\end{frame}

%%%%%%%%%%%%%
\begin{frame}
\frametitle{Diagramme de rayonnement}

Tracé avec $W = L_e = 0,4 \lambda_0$:


\begin{minipage}{0.45\textwidth}
\centering
\includegraphics[width=\textwidth]{figures/diagramme_planE.png}
\end{minipage}
\begin{minipage}{0.45\textwidth}
\centering
\includegraphics[width=\textwidth]{figures/diagramme_planH.png}
\end{minipage}

\end{frame}

%%%%%%%%%%%%%
\begin{frame}
\frametitle{Diagramme de rayonnement}
La formule $D\approx \dfrac{36400}{\theta_E^\circ\theta_H^\circ}$ donne dans ce cas $D=9\ dB$

\vspace{0.5cm}
Sachant que la conductance d'une fente est donnée par:

\[G_r=\dfrac{4\pi w^2}{3\eta\lambda^2}\]

Le modèle de la ligne sans pertes équivalente donne:

\begin{figure}
\centering\includegraphics[width=0.5\textwidth]{figures/modele_patch4.png}
\end{figure}

\hspace{3cm}$Z_e=\dfrac{1}{2.G_r}=281\ \Omega$ \hspace{1cm} (pour $w=0.4\lambda$)

\end{frame}

%%%%%%%%%%%%%
\section{Simulations}
\begin{frame}
\frametitle{Simulations 2,5D d'un patch
$0.4\lambda_0 \times  0.4 \lambda_0$}

\begin{minipage}{0.45\textwidth}
Patch $12 cm \times 12 cm$

Substrat air, épaisseur 25 mm
\end{minipage}
\begin{minipage}{0.45\textwidth}
\centering\includegraphics[width=0.7\textwidth]{figures/simulation_patch.png}
\end{minipage}

\begin{minipage}{0.45\textwidth}
Impédance = 250 $\Omega$ à la fréquence
de résonance (1 Ghz)
\end{minipage}
\begin{minipage}{0.45\textwidth}
\centering\includegraphics[width=0.7\textwidth]{figures/simulation_impedance.png}
\end{minipage}

\end{frame}

%%%%%%%%%%%%%%
\begin{frame}
\frametitle{Simulations 2,5D d'un patch
    $0.4\lambda_0 \times  0.4 \lambda_0$}

Diagramme rayonnement 2D\@:


\begin{minipage}{0.45\textwidth}
\centering\includegraphics[width=0.8\textwidth]{figures/simulation_diagramme2D_E.png}
\end{minipage}
\begin{minipage}{0.45\textwidth}
\centering\includegraphics[width=0.8\textwidth]{figures/simulation_diagramme2D_H.png}
\end{minipage}

\begin{minipage}{0.45\textwidth}
    Courants + diagramme
rayonnement 3D\@:
\end{minipage}
\begin{minipage}{0.45\textwidth}
\centering\includegraphics[width=0.8\textwidth]{figures/simulation_diagramme3D_courants.png}
\end{minipage}

\end{frame}

%%%%%%%%%%%%%%
\section{Alimentation du patch}
\begin{frame}
\frametitle{Alimentation du patch en un point
quelconque}

\begin{minipage}{0.45\textwidth}
    Alimentation au travers du
plan de masse:
\end{minipage}
\begin{minipage}{0.45\textwidth}
\centering\includegraphics[width=\textwidth]{figures/alim_plan_masse.png}
\end{minipage}

\begin{minipage}{0.45\textwidth}
    L'impédance d'entrée peut être
calculée à l'aide du modèle de la
ligne sans pertes équivalente:
\end{minipage}
\begin{minipage}{0.45\textwidth}
\centering\includegraphics[width=\textwidth]{figures/modele_alim_plan_masse.png}
\end{minipage}

\end{frame}

%%%%%%%%%%%%%%
\begin{frame}
\frametitle{Alimentation du patch en un point quelconque}
On trouve que la fréquence de résonance est inchangée, elle est telle
que:
\[l_1+l_2=\dfrac{\lambda_g}{2}-2\Delta L\]

Par contre, le choix de la position du point d'alimentation permet de
``régler'' la valeur de l'impédance à la résonance: en se
rapprochant du centre, l'impédance tend vers 0 (mais l'antenne
devient plus sélective).

\begin{minipage}{0.25\textwidth}
    Variante:
\end{minipage}
\begin{minipage}{0.65\textwidth}
\centering\includegraphics[width=\textwidth]{figures/alim_patch_cote.png}
\end{minipage}

\end{frame}

%%%%%%%%%%%%%%
\begin{frame}
\frametitle{Alimentation du patch par couplage}

\begin{figure}
\centering
\includegraphics[width=0.5\textwidth]{figures/alim_patch_slot.png}
\end{figure}

Cette méthode permet de séparer l'alimentation de l'élément rayonnant.

Inconvénient: augmentation du rayonnement arrière dû au rayonnement de la fente.

\end{frame}


\end{document}