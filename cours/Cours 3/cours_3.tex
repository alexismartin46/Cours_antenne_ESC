\documentclass[
10pt,
aspectratio=169,
]{beamer}


\usepackage{lipsum}
\usepackage{tikz}
\usepackage{bm}
\usepackage{tcolorbox}
\usepackage{beamerthemeensea}


\title{Antennes – ESC}
\subtitle{Cours 3: Antennes type ``dipôle''}
\date{\the\year}
\author{Alexis MARTIN}
%\institute{ENSEA}

\usetheme{ensea}  


\begin{document}

\begin{frame}
\titlepage\end{frame}

\begin{frame}
\tableofcontents
\end{frame}

%%%%%%%%%%%%


\section{Rayonnement d'une source quelconque de courants 
électriques}
\begin{frame}
\frametitle{Rayonnement d'une source quelconque de courants 
électriques}
\underline{Potentiel vecteur électrique}

$div(\vec{B})=0 \Rightarrow$ Il existe un vecteur $\vec{A}$ appelé \textbf{potentiel vecteur électrique} tel que: $\vec{B}=\vec{rot}(\vec{A})$

\vspace{0.5cm}
L'equation de propagation du potentiel vecteur électrique est (en utilisant la jauge de Lorentz): 
\[\Delta \vec{A}+k^2\dfrac{\partial^2\vec{A}}{\partial t^2}=-\mu\vec{J}\]

\begin{minipage}[t]{0.45\textwidth}
\vspace{0.5cm}
La solution de cette équation est:

\begin{tcolorbox}[colframe=red,width=7cm]
\centering
$\vec{A}(\vec{r})=\dfrac{\mu}{4\pi}\iiint_{sources} \vec{J}(\vec{r'})\dfrac{e^{-jkR}}{R}dV'$
\end{tcolorbox}
\end{minipage}
\begin{minipage}[t]{0.45\textwidth}
\vspace{-0.5cm}
\raggedleft\includegraphics[width=0.85\textwidth]{figures/calcul_rayonnement.png}
\vspace{-0.3cm}
\end{minipage}

\end{frame}

%%%%%%%%%%%%

\begin{frame}
\frametitle{Rayonnement d'une source quelconque de courants 
électriques}

On note $\Psi(\vec{r})=\dfrac{e^{-jk||\vec{r}||}}{||\vec{r}||}$

\begin{tcolorbox}[colframe=red,width=7cm]
\centering
$\vec{A}(\vec{r})=\dfrac{\mu}{4\pi}\iiint_{sources}\vec{J}(\vec{r'})\Psi(\vec{r}-\vec{r'})dV'$
\end{tcolorbox}

On en déduit le champ électrique rayonné:

\begin{tcolorbox}[colframe=red,width=6.5cm]
\centering
$\vec{E}=\dfrac{1}{j\omega\epsilon\mu}\left(\vec{grad}(div(\vec{A}))+k^2\vec{A}\right)$
\end{tcolorbox}

Et le champ magnétique rayonné:
\begin{tcolorbox}[colframe=red,width=4cm]
\centering
$\vec{H}=\dfrac{1}{\mu}\vec{rot}(\vec{A})$
\end{tcolorbox}
\end{frame}

%%%%%%%%%%%%

\begin{frame}
\frametitle{Rayonnement en champ lointain}
En champ lointain ($r \to \infty$), on peut faire l'approximation:

\begin{itemize}
    \item Pour les phases: $||\vec{r}-\vec{r'}|| \approx ||\vec{r}|| - \vec{e_r}.\vec{r'}$
    \item Pour les amplitudes: $||\vec{r}-\vec{r'}|| \approx ||\vec{r}||$
\end{itemize}

\vspace{-0.25cm}
\[\Psi(\vec{r}-\vec{r'}) \approx \dfrac{e^{-jk||\vec{r}||}}{||\vec{r}||}e^{jk.\vec{e_r}.\vec{r'}}\]

\begin{tcolorbox}[colframe=red,width=9cm]
\centering
$\vec{A}(\vec{r}) \approx \dfrac{\mu}{4\pi}\dfrac{e^{-jk||\vec{r}||}}{||\vec{r}||}\iiint_{sources}\vec{J}(\vec{r'})e^{jk.\vec{e_r}.\vec{r'}}dV'$
\end{tcolorbox}

On peut également montrer que:

\begin{tcolorbox}[colframe=red,width=6cm]
\centering
$\vec{E}(\vec{r}) \approx j\omega\vec{e_r}\wedge(\vec{e_r}\wedge\vec{A}(\vec{r}))$
\end{tcolorbox}

\begin{tcolorbox}[colframe=red,width=5cm]
\centering
$\vec{H}(\vec{r}) \approx \dfrac{-j.\omega}{\eta}(\vec{e_r}\wedge\vec{A}(\vec{r}))$
\end{tcolorbox}


\end{frame}



% \begin{frame} 
% \frametitle{Rayonnement d'une source quelconque de courants
% électriques}


% \begin{minipage}{0.45\textwidth}
%     \begin{figure}
%     \includegraphics[width=\textwidth]{figures/antenne_source.png}
%     \end{figure}
% \end{minipage}
% \begin{minipage}{0.45\textwidth}
% Les champs ``émis'' par une source
% quelconque $\Omega$ de courants
% volumique sont donnés par:

% \[G(\vec{r})=\dfrac{e^{-jk||\vec{r}||}}{4\pi||\vec{r}||}\]

% \end{minipage}


% \begin{tcolorbox}[colframe=red,width=11cm]
% $\vec{E}(\vec{r})=\dfrac{1}{j\omega\epsilon}\left[\vec{grad}(div())+k^2\right]\iiint_\Omega \vec{J}(\vec{r'})G(\vec{r}-\vec{r'})d^3r'$

% $\vec{H}(\vec{r})=\vec{rot}\left[\iiint_\Omega \vec{J}(\vec{r'})G(\vec{r}-\vec{r'})d^3r'\right]$
% \end{tcolorbox}

% \end{frame}

%%%%%%%%%%%%
\section{Antenne type ``doublet''}
\begin{frame}
\frametitle{Antenne type ``doublet''}

\begin{figure}
\includegraphics[width=0.5\textwidth]{figures/doublet.png}
\end{figure}


L'antenne doublet est un petit dipôle, sur lequel le courant est supposé
constant, comme proposé dans l'approximation ci-dessus.

Un dipôle petit devant la longueur d'onde ne constitue donc pas un
dipôle électrique, au sens propre, mais il en réalise une excellente
approximation.

L'antenne doublet est la première antenne réalisée expérimentalement
par Hertz.

\end{frame}

%%%%%%%%%%%%%
\begin{frame}
\frametitle{Antenne doublet électrique}

Le dipôle est dit ``miniature'' si $2.l\  <<\ \lambda$.

\vspace{0.5cm}
\begin{minipage}{0.45\textwidth}
Selon l'approximation de la ligne sans pertes équivalente, le courant
le long de l'antenne est donc donné par:
\end{minipage}
\begin{minipage}{0.45\textwidth}
\begin{figure}
\includegraphics[width=0.8\textwidth]{figures/dipole_miniature.png}
\end{figure}
\end{minipage}

\vspace{0.5cm}
\begin{minipage}{0.45\textwidth}
L'antenne étant très petite, on peut réaliser l'approximation suivante,
pour l'allure du courant sur le fil:
\end{minipage}
\begin{minipage}{0.45\textwidth}
\begin{figure}
\includegraphics[width=\textwidth]{figures/dipole_miniature2.png}
\end{figure}
\end{minipage}

\end{frame}


%%%%%%%%%%%%%

\begin{frame}
\frametitle{Rayonnement du doublet électrique}
La densité de courant sur l'antenne est donnée par
($\vec{e_z}$ est le vecteur unitaire de l'ae Oz):

\[\vec{J}(\vec{r'})=2.I.l.\delta(\vec{r'}).\vec{e_z}\]

Le potentiel vecteur électrique est donc:

\[\vec{A}(\vec{r})=\dfrac{I.l.\mu_0}{2\pi}.\Psi(\vec{r}).\vec{e_z}\]

Les champs sont alors donnés par:

\[\vec{E}(\vec{r})=\dfrac{I.l}{j2\pi\omega\epsilon}\left[\vec{grad}(div())+k^2\right]\left(\Psi(\vec{r}).\vec{e_z}\right)\]
\[\vec{H}(\vec{r})=\dfrac{I.l}{2\pi}.\vec{rot}\left[\Psi(\vec{r}).\vec{e_z}\right]\]

\end{frame}

%%%%%%%%%%%%%

\begin{frame}
\frametitle{Rayonnement du doublet électrique}
Le calcul donne:

\vspace{-0.75cm}
\[\begin{bmatrix}
    E_r \\ E_\theta \\ E_\varphi
\end{bmatrix} = \dfrac{e^{-j.k.r}}{r}.\dfrac{I.l.\eta}{j.\lambda}.
    \begin{bmatrix}
    \cos\theta\left(\dfrac{2.j}{k.r}+\dfrac{2}{k^2.r^2}\right) \\
    -\sin\theta\left(1-\dfrac{j}{k.r}-\dfrac{1}{k^2.r^2}\right)\\
    0
\end{bmatrix}\]

\[\begin{bmatrix}
    H_r \\ H_\theta \\ H_\varphi
\end{bmatrix} =j.\dfrac{e^{-j.k.r}}{r}.\dfrac{I.l}{\lambda}.
\begin{bmatrix}
    0 \\
    0 \\
    \sin\theta\left(1-\dfrac{j}{k.r}\right)
\end{bmatrix}\]

Soit l'approximation en champ lointain:

\vspace{-0.25cm}
\[\lim_{r \to \infty} \begin{bmatrix}
    E_r \\ E_\theta \\ E_\varphi
\end{bmatrix} \approx -\dfrac{e^{-j.k.r}}{r}.\dfrac{j.\eta.I.l}{\lambda}.
\begin{bmatrix}
    0 \\
    -\sin\theta\\
    0
\end{bmatrix} = \dfrac{e^{-j.k.r}}{r}.\dfrac{j.\eta.I.l}{\lambda}.\sin \theta \vec{e_\theta}
\]

\[\lim_{r \to \infty} \begin{bmatrix}
    H_r \\ H_\theta \\ H_\varphi
\end{bmatrix} \approx j.e^{-j.k.r}.\dfrac{1}{2.r}.\dfrac{2.I.l}{\lambda}.
\begin{bmatrix}
    0 \\
    0\\
    \sin\theta
\end{bmatrix} = j.e^{-j.k.r}.\dfrac{1}{2.r}.\dfrac{2.I.l}{\lambda}.\sin \theta \vec{e_\varphi}
\]

\end{frame}

%%%%%%%%%%%%%

\begin{frame}
    \frametitle{Rayonnement du doublet électrique}

%\vspace{-0.5cm}
\begin{minipage}{0.35\textwidth}
    On retiendra:
\end{minipage}
\begin{minipage}{0.6\textwidth}
\begin{tcolorbox}[colframe=red,width=7cm]
    $\vec{E}(\vec{r})=\dfrac{j.\eta.I.l_{tot}}{2\lambda}\dfrac{e^{-j.k.r}}{r}.\sin\theta.\vec{e_\theta}$
    
    $\vec{H}(\vec{r})=\dfrac{j.I.l_{tot}}{2\lambda}\dfrac{e^{-j.k.r}}{r}.\sin\theta.\vec{e_\varphi}$
\end{tcolorbox}
\end{minipage}

\begin{minipage}{0.35\textwidth}
Vecteur de Poynting radial: 
\end{minipage}
\begin{minipage}{0.6\textwidth}
\begin{tcolorbox}[colframe=red,width=8cm]
    $\vec{P}=\dfrac{1}{2}\vec{E}\wedge\vec{H}^* = \dfrac{\eta}{2}{\left(\dfrac{I}{2.r}\right)}^2.{\left(\dfrac{l}{\lambda}\right)}^2 .\sin^2\theta.\vec{e_r}$
\end{tcolorbox}
\end{minipage}

\begin{minipage}{0.35\textwidth}
Onde localement plane:
\end{minipage}
\begin{minipage}{0.6\textwidth}
\begin{tcolorbox}[colframe=red,width=4cm]
    $\vec{H}(\vec{r})=\dfrac{1}{\eta}(\vec{e_r}\wedge\vec{E})$

    $\vec{E}(\vec{r})=\eta (\vec{H} \wedge \vec{e_r})$
\end{tcolorbox}
\end{minipage}

\begin{minipage}{0.35\textwidth}
Caractéristique de rayonnement:
\end{minipage}
\begin{minipage}{0.6\textwidth}
 \begin{tcolorbox}[colframe=red,width=4cm]
    $r(\theta,\varphi)=\sin^2\theta$
\end{tcolorbox}
\end{minipage}

\end{frame}

%%%%%%%%%%%%%
\begin{frame}
\frametitle{Rayonnement du doublet électrique}

\begin{minipage}{0.45\textwidth}
    \begin{figure}
    \includegraphics[width=\textwidth]{figures/Rayonnement_doublet.png}
\end{figure}
\end{minipage}
\begin{minipage}{0.45\textwidth}
\begin{figure}
    \includegraphics[width=\textwidth]{figures/Rayonnement_doublet2.png}
\end{figure}
\end{minipage}

\vspace{0.5cm}
\begin{center}
Directivité: $D=\dfrac{3}{2}$
\end{center}

\end{frame}

%%%%%%%%%%%%%
\begin{frame}
    \frametitle{Rayonnement du doublet électrique}
En champ lointain, les lignes de champ électrique
sont des cercles portés par $\vec{e_\theta}$.

\vspace{0.5cm}
La polarisation est linéaire, contenue dans un plan vertical:

\begin{figure}
    \includegraphics[width=0.4\textwidth]{figures/champ_doublet.png}
\end{figure}

\end{frame}
%%%%%%%%%%%%%%

\begin{frame}
\frametitle{Impédance d'entrée du doublet électrique}
\begin{minipage}{0.45\textwidth}
$R_r=\dfrac{80\pi^2.l_{tot}^2}{\lambda^2}$

$X = - R_C \cot\left(\dfrac{\pi l_{tot}}{\lambda}\right)$ (\textbf{avec un modèle de ligne sans pertes})
\end{minipage}
\begin{minipage}{0.45\textwidth}
\begin{figure}
\includegraphics[width=\textwidth]{figures/impedance_doublet.png}
\end{figure}
\end{minipage}

A.N. avec $l_{tot}=\dfrac{\lambda}{20}$ et $R_C=200\Omega$:

$R_r=1.9 \Omega$ et $X=-615 \Omega$

$\rightarrow$ Le doublet électrique est très dificile à adapter, et a certainement un mauvais rendement.

\end{frame}

%%%%%%%%%%%%%
\section{Antenne type ``dipôle''}
\begin{frame}
\frametitle{Antenne ``type dipôle''}

\begin{minipage}{0.45\textwidth}
Un signal électrique est appliqué sur un fil de
longueur $2\ l$ selon le montage:
\end{minipage}
\begin{minipage}{0.45\textwidth}
\begin{figure}
\includegraphics[width=0.5\textwidth]{figures/dipole.png}
\end{figure}
\end{minipage}

Il est possible de calculer une valeur approchée du courant sur le fil
selon le modèle de la ligne sans pertes équivalente:

\begin{figure}
\includegraphics[width=0.7\textwidth]{figures/dipole_courant.png}
\end{figure}

\end{frame}

%%%%%%%%%%%%%
\begin{frame}
\frametitle{Antenne ``type dipôle''}
On a alors:

\[V(z)=V_{\max}\cos\left(\beta (l-z)\right)\]
\[I(z)=j.\dfrac{V_{\max}}{R_c}\cos\left(\beta (l-z)\right)\]

\begin{minipage}{0.45\textwidth}
\begin{figure}
\includegraphics[width=\textwidth]{figures/dipole_courant2.png}
\end{figure}
\end{minipage}
\begin{minipage}{0.45\textwidth}
\begin{figure}
\includegraphics[width=\textwidth]{figures/dipole_tension.png}
\end{figure}
\end{minipage}

\end{frame}

%%%%%%%%%%%%%%
\begin{frame}
\frametitle{Antenne ``type dipôle''}

\begin{minipage}{0.45\textwidth}
$V_{\max}$ varie avec la fréquence selon:

\[V_{\max}=\dfrac{V_g}{\sqrt{\cos^2\left(\dfrac{\omega l}{v}\right) + \dfrac{R_g^2}{R_c^2}\sin^2\left(\dfrac{\omega l}{v}\right)}}\]

\vspace{1cm}

Le cas des dipôles correspond
au 1er cas: celui où $R_g << R_c$
\end{minipage}
\begin{minipage}{0.45\textwidth}
\begin{figure}
\includegraphics[width=0.9\textwidth]{figures/dipole_tension_max.png}
\end{figure}
\end{minipage}

\end{frame}

%%%%%%%%%%%%%%
\begin{frame}
\frametitle{Fréquence de raisonnance du dipôle}
Les fréquences de résonance des antennes type dipôle s'obtiennent pour:

\[\dfrac{\omega l}{v}=\dfrac{\pi}{2},\ \dfrac{3\pi}{2},\ \dfrac{5\pi}{2}\ \dots \]


Soit: $l_{tot} = \dfrac{\lambda}{2},\  \dfrac{3\lambda}{2},\ \dfrac{5\lambda}{2},\ \dots$

\vspace{0.5cm}
Le premier cas est le cas le plus courant, cette antenne est appelée:

\begin{center}
\begin{tcolorbox}[colframe=red,width=6cm]
    \centering\textbf{Dipôle demi-onde}
\end{tcolorbox}
\end{center}

\end{frame}

%%%%%%%%%%%%%%
\begin{frame}
\frametitle{Dipôle demi-onde}
Selon l'approximation de la ligne sans pertes équivalente, le courant
le long de l'antenne est donc donné par:

\begin{minipage}{0.45\textwidth}
    Pour $|z| \leq \dfrac{\lambda}{4}$:
\[I(z)=I(0).\cos\left(\dfrac{2\pi z}{\lambda}\right)\]

Et l'impédance d'entrée:
\[Z_e=-j.R_c\cot\left(\dfrac{\pi l_{tot}}{\lambda}\right)\]

\end{minipage}
\begin{minipage}{0.45\textwidth}
\begin{figure}
\includegraphics[width=0.9\textwidth]{figures/courant_dipole.png}
\end{figure}
\end{minipage}

\end{frame}

%%%%%%%%%%%%%%
\begin{frame}
\frametitle{Rayonnement du dipôle demi-onde}

\begin{minipage}{0.45\textwidth}
\[\vec{E}(\vec{r})=j\dfrac{\eta I(0)}{2\pi r}e^{-j.k.r}\dfrac{\cos\left(\dfrac{\pi}{2}\cos\theta\right)}{\sin\theta}\vec{u_\theta}\]

\[U(\theta,\varphi)=\dfrac{\eta|I(0)|^2}{8\pi^2}.\dfrac{\cos^2\left(\dfrac{\pi}{2}\cos\theta\right)}{\sin^2\theta}\]

\[r(\theta,\varphi)=\dfrac{\cos^2\left(\dfrac{\pi}{2}\cos\theta\right)}{\sin^2\theta}\]

\vspace{0.5cm}
\centering Directivité = 1.64
\end{minipage}
\begin{minipage}{0.45\textwidth}
\begin{figure}
\includegraphics[width=0.7\textwidth]{figures/dipole_radiation2D.png}
\end{figure}

\vspace{-1cm}
\begin{figure}
\includegraphics[width=0.5\textwidth]{figures/dipole_radiation3D.png}
\end{figure}
\end{minipage}

\end{frame}

%%%%%%%%%%%%%%
\begin{frame}
\frametitle{Impédance d'entrée du dipôle demi-onde}
\begin{minipage}{0.45\textwidth}
$R_r=73\Omega$ 

\vspace{0.5cm}
$X=0 \Omega$ (à la fréquence de raisonnance)
\end{minipage}
\begin{minipage}{0.45\textwidth}
\begin{figure}
\includegraphics[width=\textwidth]{figures/impedance_doublet.png}
\end{figure}
\end{minipage}

\begin{minipage}{0.45\textwidth}
Variations de X en fonction de la fréquence:

\vspace{0.5cm}
$X = - R_c \cot\left(\dfrac{2\pi l}{\lambda}\right)$

\vspace{0.5cm}
$R_c = 200 \ln\left(\dfrac{l}{a}\right)$

\vspace{0.5cm}
$a$ = rayon du fil
\end{minipage}
\begin{minipage}{0.45\textwidth}
\begin{figure}
\includegraphics[width=0.9\textwidth]{figures/impedance_dipole.png}
\end{figure}
\end{minipage}

\end{frame}
%%%%%%%%%%%%%%
\begin{frame}
\frametitle{Simulation 3D d'un dipôle demi-onde}

\vspace{-0.25cm}
\centering Courant sur le fil:

\begin{minipage}{0.45\textwidth}
    \begin{figure}
    \includegraphics[width=0.6\textwidth]{figures/simu_courant_dipole.png}
    \end{figure}
\end{minipage}
\begin{minipage}{0.45\textwidth}
    \begin{figure}
    \includegraphics[width=0.6\textwidth]{figures/simu_courant_dipole2.png}
    \end{figure}
\end{minipage}

Digramme de rayonnement:

\begin{minipage}{0.45\textwidth}
    \begin{figure}
    \includegraphics[width=0.6\textwidth]{figures/simu_diagramme_dipole_2D.png}
    \end{figure}
\end{minipage}
\begin{minipage}{0.45\textwidth}
    \begin{figure}
    \includegraphics[width=0.6\textwidth]{figures/simu_diagramme_dipole_3D.png}
    \end{figure}
\end{minipage}

\end{frame}
%%%%%%%%%%%%%%

\begin{frame}
\frametitle{Alimentation d'un dipôle demi-onde}
\vspace{-1cm}
\begin{minipage}{0.45\textwidth}
    
    Alimentation incorrecte:

\end{minipage}
\begin{minipage}{0.45\textwidth}
    \begin{figure}
    \includegraphics[width=0.5\textwidth]{figures/dipole_alimentation_incorrecte.png}
\end{figure}
\end{minipage}

\vspace{-0.5cm}
Alimentation par un balun:

\begin{minipage}{0.3\textwidth}
    \begin{figure}
    \includegraphics[width=0.8\textwidth]{figures/dipole_balun.png}
    \end{figure}
\end{minipage}
\begin{minipage}{0.3\textwidth}
    \begin{figure}
    \includegraphics[width=0.8\textwidth]{figures/dipole_balun2.png}
    \end{figure}
\end{minipage}
\begin{minipage}{0.3\textwidth}
    \begin{figure}
    \includegraphics[width=0.8\textwidth]{figures/dipole_balun3.png}
    \end{figure}
\end{minipage}

\vspace{-0.5cm}

\begin{flushright}
{\small Wikipédia}
\end{flushright}
\end{frame}

%%%%%%%%%%%%%%

\section{Effet image}
\begin{frame}
\frametitle{Effet image}

Au niveau d'un plan parfaitement conducteur, le champ électrique est normal au plan, et le champ magnétique est tangent.

Il en résulte que le champ créé par une charge électrique ponctuelle au dessus d'un plan de masse est équivalent à celui créé par cette charge et son image, qui est une charge de même valeur mais de signe opposé, placée symétriquement par rapport au plan de masse. 
\begin{figure}
\includegraphics[width=0.8\textwidth]{figures/effet_image_charges.png}
\end{figure}

\end{frame}

%%%%%%%%%%%%%%%

\begin{frame}
\frametitle{Effet image}

\begin{minipage}{0.35\textwidth}
Il en va de même pour des charges en mouvement, comme dans le cas d'un dipôle électrique placé au-dessus d'un plan de masse.
\end{minipage}
\begin{minipage}{0.55\textwidth}
\begin{figure}
\includegraphics[width=\textwidth]{figures/effet_image_dipole.png}
\end{figure}
\end{minipage}

\begin{minipage}{0.35\textwidth}
On peut donc généraliser l'effet image à toute répartition de courants au-dessus d'un plan de masse.
\end{minipage}
\begin{minipage}{0.55\textwidth}
\begin{figure}
\includegraphics[width=\textwidth]{figures/effet_image_courant.png}
\end{figure}
\end{minipage}

\end{frame}

%%%%%%%%%%%%%%%

% \begin{frame}
% \frametitle{Propriétés de symétrie des champs}
% Les propriétés de symétrie des équations de Maxwell impliquent que,
% lors d'une symétrie par rapport à un plan:

% \begin{itemize}
%     \item Le champ électrique a les mêmes propriétés de symétrie que
% les courants électriques (courants symétriques $\Leftrightarrow$ E symétrique, ou
% courants antisymétriques $\Leftrightarrow$ E antisymétrique).

%     \item Le champ magnétique a des propriétés de symétrie opposées à
% celles des courants électriques (courants symétriques $\Leftrightarrow$ H
% antisymétrique, ou courants antisymétriques $\Leftrightarrow$ H symétrique).
% \end{itemize}

% Donc, si les courants sont antisymétriques par rapport à un plan, le
% champ électrique sera aussi antisymétrique, ce qui implique que la
% champ électrique est normal à ce plan.

% \vspace{0.5cm}
% \textit{L'effet image résulte de cette remarque.}

% \end{frame}
%%%%%%%%%%%%%%

% \begin{frame}
% \frametitle{Effet image}

% \begin{minipage}{0.45\textwidth}
% Soit une répartition de champs vérifiant les équations de Maxwell:
% \end{minipage}
% \begin{minipage}{0.45\textwidth}
% \begin{figure}
% \includegraphics[width=0.7\textwidth]{figures/effet_image.png}
% \end{figure}
% \end{minipage}

% \begin{minipage}{0.45\textwidth}
% Par symétrie, la répartition suivante vérifie aussi les équations de
% Maxwell:
% \end{minipage}
% \begin{minipage}{0.45\textwidth}
% \begin{figure}
% \includegraphics[width=0.7\textwidth]{figures/effet_image2.png}
% \end{figure}
% \end{minipage}

% \begin{minipage}{0.45\textwidth}
% Ainsi que la superposition des deux répartitions ci-dessus:
% \end{minipage}
% \begin{minipage}{0.45\textwidth}
% \begin{figure}
% \includegraphics[width=0.7\textwidth]{figures/effet_image_sup.png}
% \end{figure}
% \end{minipage}

% \end{frame}
% %%%%%%%%%%%%%%

% \begin{frame}
% \frametitle{Effet image}
% Sur le plan de symétrie ci-dessus, E est normal, et H tangentiel.

% Si on ``métallise'' le plan de symétrie, les champs restent donc
% inchangés des deux côtés.

% Les deux situations suivantes donnent donc les mêmes champs du côté
% des $z > 0$:

% \begin{itemize}
% \item \begin{minipage}{0.45\textwidth}
% Sans plan de masse
% \end{minipage}\begin{minipage}{0.45\textwidth}
%     \begin{figure}
%     \includegraphics[width=0.8\textwidth]{figures/effet_image_sup.png}
%     \end{figure}
% \end{minipage}
% \item \begin{minipage}{0.45\textwidth}
%     Avec plan de masse (plan $z=0$)
%     \end{minipage}\begin{minipage}{0.45\textwidth}
%         \begin{figure}
%         \includegraphics[width=0.8\textwidth]{figures/effet_image.png}
%         \end{figure}
%     \end{minipage}
% \end{itemize}

% \end{frame}
%%%%%%%%%%%%%%

\begin{frame}
\frametitle{Analogie optique}

Pour un observateur placé au-dessus du plan de masse, les
deux scènes suivantes sont équivalentes:

\begin{figure}
\includegraphics[width=\textwidth]{figures/effet_image_optique.png}
\end{figure}

\end{frame}
%%%%%%%%%%%%%%

\begin{frame}
\frametitle{Effet image ``électrique''}
Les champs au-dessus du plan de masse sont identiques
dans les deux situation suivantes:

\begin{figure}
    \includegraphics[width=\textwidth]{figures/effet_image_elec.png}
\end{figure}

\end{frame}
%%%%%%%%%%%%%%

\section{Dipôle quart d'onde}
\begin{frame}
\frametitle{Dipôle quart d'onde}
Un dipôle quart d'onde se comporte comme un demi-onde, pour le
courant et les champs au-dessus du plan de masse:

\begin{figure}
    \includegraphics[width=0.9\textwidth]{figures/dipole_quart_onde.png}
\end{figure}

\end{frame}
%%%%%%%%%%%%%%

\begin{frame}
\frametitle{Dipôle quart d'onde}
\begin{center}
\begin{minipage}{0.22\textwidth}
    \includegraphics[width=\textwidth]{figures/dipole_quart_onde2.png}
\end{minipage}
\begin{minipage}{0.35\textwidth}
    \includegraphics[width=\textwidth]{figures/dipole_quart_onde3.png}
\end{minipage}
\begin{minipage}{0.22\textwidth}
    \includegraphics[width=\textwidth]{figures/dipole_quart_onde4.png}
\end{minipage}
\end{center}

$R_r=37\ \Omega$

$R_c=60 \ln\left(\dfrac{l}{a}\right)$

$D=1.64$ ou $D=3.28$ selon que l'on considère
l'espace entier ou bien un 1/2 espace.

\end{frame}

\end{document}