\documentclass[a4paper,11pt]{article}
\usepackage[]{amsmath,amsfonts,theorem,amssymb,stmaryrd}
\usepackage[french]{babel}
\usepackage[utf8]{inputenc}% Pour saisir directement les caracteres accentues codage utf-8
\usepackage{lastpage} % Pour pouvoir insérer le numéro de la dernière page
\usepackage{psfrag}   %pour pouvoir modifier du texte sur les figures
\usepackage{hyperref}   % pour creer des liens que peut suivre Acrobat reader
\usepackage{alterqcm} %pour les qcm
\setlength{\unitlength}{1mm} %l'unité par défaut est le mm
\usepackage[final]{pdfpages} 
\usepackage{enumitem}
\usepackage{multicol}% multicolonnes
\usepackage{hyperref}

%--------------------------------------Pour les figures-----------------------------%
\usepackage{float}
\usepackage{graphicx}
\usepackage{subfigure} % pour les sous-figures
\DeclareGraphicsExtensions{.eps} % tous les fichiers graphiques sont
                                 % par defaut des fichiers .eps
%-----------------------------------------------------------------------------------%



%%%%%%%%%%%%%%%%%%%%%%%%%%%%%%% MISE EN PAGE %%%%%%%%%%%%%%%%%%%%%%%%%%%%%%%%%%%%%%%%%%%%%%%%%%%

\headheight=70pt   % hauteur de l'entete
\textwidth=17cm    % largeur du texte hors note en marge
\oddsidemargin=-20pt % marge gauche réduite à 1 pouce
\textheight=24.5cm   % longueur utile de la page
\topmargin=-60pt   % marge haute négative (0pt correspond à une marge de 1 pouce)
\marginparwidth=0pt%
\marginparsep=0pt  %
\headsep=20pt      % séparation 20 points entre entête et texte
\footskip=30pt % iden séparation pied de page


%%%%%%%%%%%%%%%%%%%%%%%%%%%%%%%%%%%%%%%%%%%%%%%%%%%%%%%%%%%%%%%%%%%%%%%%%%%%%%%%%%%%%%%%%%%%%%%

%----------------------Définition des entetes et pieds de pages--------------------%
\usepackage{fancyhdr} % pour agir sur les entetes et pieds de pages
\lhead{\includegraphics*[width=1.5cm]{fig/logoEnseaCoul}}
\chead{\large\textsl{Contrôle de synthèse \the\year\\ Antennes}}
\rhead{Nom:\ \ \ \ \ \ \ \ \ \ \ \ \ \ \ \ \ \ \ \ \ \ \ \ \ \ \ \\ Prénom :\ \ \ \ \ \ \ \ \ \ \ \ \ \ \ \ \ \ \ \ \ }
\lfoot{\small\textsl{A. Martin}}
\cfoot{\small\textsl{Décembre \the\year}}
\rfoot{\normalsize\textsl{\thepage /\pageref{LastPage}}}
\renewcommand{\footrulewidth}{0.4pt} % pour l'epaisseur du trait en bas de page
\pagestyle{fancy}
%----------------------------------------------------------------------------------%

\begin{document}
\vspace{30pt}

%\vspace{10pt}


%%%%%%%%%%%%%%%%%%%%%%%%%%%%%%%%%%%
%%%%%%%%    SUJET 2025    %%%%%%%%%
%%%%%%%%%%%%%%%%%%%%%%%%%%%%%%%%%%%
\textbf{2h. Documents et calculatrice autorisés.}

\section{Questions de cours}

\underline{\textit{Toutes les questions de cette partie sont indépendantes.}}

\begin{enumerate}
	\item Voici l'impédance d'entrée d'une antenne patch ($|S_{11}|$ à gauche et abacque de Smith à droite).  
	\begin{center}
		\includegraphics[width=0.5\textwidth]{fig/QCM_S11_dB.png}
		\includegraphics[width=0.3\textwidth]{fig/QCM_S11_smith.png}
	\end{center}
	La fréquence de résonnance de cette antenne est:
	\begin{itemize}
		\item[$\square$] 2.459 GHz
		\item[$\square$] 2.483 GHz
	\end{itemize}

	\item Le résultat de la simulation d'une antenne patch est présenté ci-dessous (courbe du gain dans le plan E en dBi).
	
	\begin{center}
		\includegraphics[width=0.9\textwidth]{fig/QCM_farfield.png}
	\end{center}

	L'ouverture à mi-puissance vaut:
	\begin{itemize}
		\item[$\square$] 0$^{\circ}$
		\item[$\square$] 38$^{\circ}$
		\item[$\square$] 76$^{\circ}$
		\item[$\square$] 84$^{\circ}$
		\item[$\square$] 152$^{\circ}$
	\end{itemize}

	\newpage
	\item On considère l'antenne patch ci-dessous. Le plan E est le plan:
	
	\begin{minipage}{0.3\textwidth}
		\begin{itemize}
			\item [$\square$] (xOz)
			\item [$\square$] (yOz)
			\item [$\square$] (xOy)
		\end{itemize}
	\end{minipage}
	\begin{minipage}{0.4\textwidth}
		\includegraphics[width=\textwidth]{fig/patch.png}
	\end{minipage}

	\item Pour l'antenne patch précédente, le plan H est le plan:
	
	\begin{minipage}{0.3\textwidth}
		\begin{itemize}
			\item [$\square$] $\varphi=0^\circ$
			\item [$\square$] $\varphi=90^\circ$
			\item [$\square$] $\theta=0^\circ$
			\item [$\square$] $\theta=90^\circ$
		\end{itemize}
	\end{minipage}
	
	
	
	\item On considère une antenne isotrope rayonnant une puissance de 0dBm à 2.4GHz. La puissance reçue par une antenne de réception de gain 6 dBi située à 10 m de distance dans l'axe de l'antenne est (on suppose les antennes parfaitement adaptées et avec la même polarisation):
	
	\begin{picture}(160,40)
		\framebox(160,40){}
	\end{picture}

	\item Quel est l'avantage d'utiliser une antenne à polarisation circulaire à l'émission et une antenne à polarisation linéaire à la réception (par rapport à 2 antennes à polarisation linéaire)?
	
	\begin{picture}(160,40)
		\framebox(160,40){}
	\end{picture}

	\item On souhaite réaliser une antenne monopôle fonctionnant à 868 MHz (application LoRa). Quelle doit être la hauteur physique de cette antenne?
	
	\begin{picture}(160,30)
		\framebox(160,30){}
	\end{picture}

	\newpage
	\item Quel est l'avantage d'une antenne cornet par rapport à un réseau d'antennes imprimées?
	\begin{itemize}
		\item [$\square$] Le cornet est plus compact
		\item [$\square$] Le cornet peut transmettre plus de puissance
		\item [$\square$] Le cornet est plus facile à intégrer dans un circuit imprimé
		\item [$\square$] Le cornet a un gain plus élevé
		\item [$\square$] Le cornet est plus léger
		\item [$\square$] Le cornet est moins cher
	\end{itemize}

	\item La figue ci-dessous représente le champ électrique au niveau de l'ouverture d'un cornet. Le plan E correspond au plan:
	
	\begin{minipage}{0.3\textwidth}
		\begin{itemize}
			\item [$\square$] (xOz)
			\item [$\square$] (yOz)
			\item [$\square$] (xOy)
		\end{itemize}
	\end{minipage}
	\begin{minipage}{0.5\textwidth}
		\includegraphics[width=\textwidth]{fig/Champ_cornet.png}
	\end{minipage}
	



\end{enumerate}

%%%%%%%%%%%%%%%%%%%%%%%%%%%%%%
%%%%%%%% Exercice 1  %%%%%%%%%
%%%%%%%%%%%%%%%%%%%%%%%%%%%%%%
\newpage
\section{Exercice : Antenne miniature}
On se propose d'étudier l'antenne miniature 7488920245 de Würth Elektroniks, dont une partie de la documentation technique est disponible en annexe. 

\begin{enumerate}
	\item Quelle est la bande passante de cette antenne ? On utilisera le critère classique pour les antennes.
	
	\begin{picture}(160,20)
		\framebox(160,20){}
	\end{picture}

	\item A partir de vos connaissances, justifier la forme du diagramme de rayonnement de l'antenne.
	
	\begin{picture}(160,30)
		\framebox(160,30){}
	\end{picture}

	\item Cette antenne est à polarisation:
	\begin{itemize}
        \item[$\square$] linéaire
        \item[$\square$] circulaire
    \end{itemize}

	\vspace{0.5cm}
	On souhaite réaliser un réseau d'antennes \textit{end-fire} utilisant cette antenne miniature (voir figure ci-dessous). On souhaite donc que ce réseau rayonne dans la direction $\vec{y}$ (les antennes sont placées sur l'axe $\vec{y}$).
	
	\begin{center}
		\includegraphics[width=0.3\textwidth]{fig/ceramic_antenna_reseau_base.png}
	\end{center}

	\vspace{-0.5cm}
	\item Quelle doit être l'orientation des antennes pour que le réseau puisse rayonner dans la direction souhaitée ? (Justifier)
	
	\includegraphics[width=0.4\textwidth]{fig/ceramic_antenna_reseau.png}\hspace{2cm}
    \includegraphics[width=0.4\textwidth]{fig/ceramic_antenna_reseau2.png}
    
	\vspace{-0.5cm}
	\begin{center}
		\textbf{A} \hspace{8.5cm} \textbf{B}
	\end{center}

	\begin{picture}(160,35)
		\framebox(160,35){}
	\end{picture}

	\item Quel doit être le déphasage $\Psi$ entre chaque antenne pour que le réseau rayonne dans la direction souhaitée ? (Justifier)
	
	\begin{picture}(160,30)
		\framebox(160,30){}
	\end{picture}

	\item Tracer l'allure de la caractéristique de rayonnement dans les 3 plans de coupes principaux (plans $\varphi=0^\circ$, $\varphi=90^\circ$ et $\theta=90^\circ$) du réseau formé par ces 3 antennes avec le déphasage calculé, en coordonnées polaires.
	
	\begin{minipage}{0.32\textwidth}
		\includegraphics[width=\textwidth]{fig/polar.png}
		
		\centering Plan $\varphi=0^\circ$
	\end{minipage}
	\begin{minipage}{0.32\textwidth}
		\includegraphics[width=\textwidth]{fig/polar.png}
		
		\centering Plan $\varphi=90^\circ$
	\end{minipage}
	\begin{minipage}{0.32\textwidth}
		\includegraphics[width=\textwidth]{fig/polar.png}
		
		\centering Plan $\theta=90^\circ$
	\end{minipage}


\end{enumerate}

%%%%%%%%%%%%%%%%%%%%%%%%%%%%%%
%%%%%%%% Exercice 2  %%%%%%%%%
%%%%%%%%%%%%%%%%%%%%%%%%%%%%%%
\newpage
\section{Exercice: Réseau d'antennes patchs}

On souhaite réaliser une liaison point à point à 24 GHz (uplink). On utilise pour cela un réseau d'antennes patchs imprimées. On donne la caractéristique de rayonnement dans les plans E et H.

\begin{center}
\begin{minipage}{0.7\textwidth}
	\includegraphics[width=\textwidth]{fig/array_factor_N6_08lambda_E.png}
	
	\centering Caractéristique de rayonnement dans le plan E
\end{minipage}

\begin{minipage}{0.7\textwidth}
	\includegraphics[width=\textwidth]{fig/array_factor_N8_07lambda_H.png}
	
	\centering Caractéristique de rayonnement dans le plan H
\end{minipage}		
\end{center}

\begin{enumerate}
	\item Quel est le niveau de lobe secondaire dans le plan E par rapport au lobe principal? (en dB)
	
	\begin{picture}(160,20)
		\framebox(160,20){}
	\end{picture}

	\item Sachant qu'il y a 6 antennes dans le plan E, quelle est la distance entre chaque antenne dans cette direction?

	\begin{picture}(160,40)
		\framebox(160,40){}
	\end{picture}

	\item Sachant que la distance entre antennes dans le plan H est de 8.7mm, combien y a-t-il d'antennes dans cette direction? 

	\begin{picture}(160,40)
		\framebox(160,40){}
	\end{picture}

	\item Quelle est la directivité de l'antenne? (en dBi)

	\begin{picture}(160,40)
		\framebox(160,40){}
	\end{picture}

	\item Sachant que l'efficacité de l'antennne est de 80\%, quel est le gain de l'antenne? (en dBi)

	\begin{picture}(160,30)
		\framebox(160,30){}
	\end{picture}

	\item La surface de l'antenne est de 70mm x 79mm. Quelle est la valeur du coefficient d'ouverture de l'antenne?
	
	\begin{picture}(160,30)
		\framebox(160,30){}
	\end{picture}

	\item On souhaite réaliser une liaison point à point entre deux antennes identiques situées à 200 m l'une de l'autre. La puissance émise par l'antenne d'émission est de 10 dBm. Quelle est la puissance reçue par l'antenne de réception? (en dBm)
	
	On suppose que les antennes sont parfaitement adaptées, avec la même polarisation, et que l'environnement de propagation est libre (\textit{Line of Sight}).

	\begin{picture}(160,30)
		\framebox(160,30){}
	\end{picture}

\end{enumerate}


\newpage
\subsection*{Annexe: Documentation technique de l'antenne miniature 7488920245}
\includegraphics[width=0.89\textwidth]{fig/ceramic_antenna_s11.png}

\includegraphics[width=0.89\textwidth]{fig/ceramic_antenna_diagram.png}

\includegraphics[width=0.89\textwidth]{fig/ceramic_antenna_diagram2.png}

%\subsection*{Annexe: Documentation technique de l'antenne patch PA2400-A}


\end{document}
